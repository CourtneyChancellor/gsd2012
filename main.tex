% Guide de Survie du Doctorant (à Nantes)

\documentclass[11pt]{report}

%\usepackage{hyperref}

\usepackage[english,french]{babel}
\usepackage[utf8]{inputenc}
\usepackage[T1]{fontenc}
\usepackage{ifthen}

%\usepackage{amsmath}  % Maths
%\usepackage{amsfonts} % Maths
%\usepackage{amssymb}  % Maths
%\usepackage{stmaryrd} % Maths (crochets doubles)

\usepackage{url}     % Mise en forme + liens pour URLs
\usepackage{array}   % Tableaux évolués
\usepackage{eurosym} % Symbole euro €

\usepackage{comment}

\usepackage[hmargin=3cm,vmargin=2.5cm]{geometry}

\begin{comment}
\usepackage{tikz}
\newdimen\pgfex
\newdimen\pgfem
\usetikzlibrary{arrows,shapes,shadows,scopes}
\usetikzlibrary{positioning}
\usetikzlibrary{matrix}
\usetikzlibrary{decorations.text}
\usetikzlibrary{decorations.pathmorphing}
\end{comment}

% Commandes À FAIRE
\usepackage{color} % Couleurs du texte
\definecolor{darkred}{rgb}{0.5,0,0}
%\newcommand{\afaire}[1]{\textcolor{red}{[À FAIRE~: #1]}}
\newcommand{\trad}{\textcolor{blue}{\textbf{[Traduction à faire]}}\par}
\newcommand{\todo}[1]{\textcolor{red}{\textbf{[#1]}}}


% Id est
\newcommand{\ie}{\textit{i.e.} }



\title{Guide de Survie du Doctorant}

\author{AED Centrale Nantes}



\begin{document}
    \maketitle
% TODO: Page de titre

\setcounter{page}{0}
\tableofcontents

\newpage
\chapter*{Introduction}
\todo{Le blabla du président ici}

Maxime FOLSCHETTE is awesome

%Blabla 2 (suite à une pb de synchronisation depuis la version locale)
Dear PhD students,

you now arrived on the campus of the big and prestigious École Centrale de Nantes as a PhD student!
If you are reading this, it means that you're already on the scene \todo{(?)} and you're about to begin.
We do not know yet who you are, where you come from and if what we wrote here is useful.
Maybe you're already familiar with the school and its surroundings?
But many of you know nothing about this school, the life in the laboratory, in Nantes or even life in France for some foreigners.
We wrote this little booklet for these people.
It will introduce you to the practical life in and outside the campus.
What will happen to you? Are you going out? This document is meant to help you quickly.
It was conducted by members of the Association des Étudiants en Doctorat (AED), which is an association dedicated to PhD students on the Ecole Centrale de Nantes campus (which you may join later, why not).
Without necessarily get there, it is a living document, which asks only to be improved. \todo{(?)}
Time that the first question \todo{(???)}, if you have questions this booklet does not answer, do not hesitate to contact us, we will do our best to answer them!

\section*{The AED (Association des Étudiants en Doctorat) (PhD Students Association)}
\begin{quotation}
The thesis takes time, changing its author into a hybrid creature, half-student, half-adult, stuck between two ages.
It also consumes space by stacking books, cards and drafts. It abrades the privacy. It gives headaches.
She is also the heart, \todo{(?)} it creates emotional involvement [\dots] It's also the object of all his troubles
\end{quotation}
\todo{Citation de qui ?}

This is why it is necessary to fight all together, united, compacted against it.
To this end, we have decided to create an association of PhD students on the campus of the École Centrale.
The goals of the association are to:
\begin{itemize}
\item enable doctoral students of different teams, different labs to exchange and gather.
\item organize scientific and cultural events throughout the year (PhD students workshops, karting, pool, tree climbing, parties, \dots)
\item information towards the PhD students of the school
\end{itemize}
Of course, all PhD students are welcome in the association.

%\chapter{L'École Centrale de Nantes et le campus universitaire}
\chapter{The Ecole Centrale de Nantes and campus}

%\section{L'enseignement supérieur français}\trad
\section{French higher education}

%Comme pour les autres pays européens, l'enseignement supérieur est basé sur le système « LMD » : Licence Master Doctorat.
%En France, il n'y a pas que les universités, mais il existe également des Grandes Écoles, dites écoles d'ingénieurs, comme l'École Centrale Nantes.
%Pour débuter un doctorat en France, on peut avoir suivi soit un cursus universitaire (en passant donc par un master recherche) soit venir d'une école d'ingénieurs.

As in other European countries, higher education is based on the "LMD" Bachelor Master Doctorate.
In France, there is not only universities, but there are also Grandes Ecoles, known engineering schools such as the Ecole Centrale Nantes.
To start a PhD in France, can be followed by either a university degree (thus passing by a research master) or come from an engineering school.

\section{Tout sur l'École Centrale de Nantes}\trad
\subsection{L'école en quelques chiffres}\trad
L'école a été fondée il y a plus de 90 ans.
D'abord appelée l'IPO (Institut Polytechnique de l'Ouest depuis 1919), puis l'ENSM (École Nationale de Mécanique à partir de 1947), elle est rattachée au groupe des Écoles Centrales en 1991.
Elle était initialement basée en centre ville, elle est arrivée sur le campus du Tertre en 1977.
\todo{Vérifier ces chiffres}
\paragraph{Les gens}
\begin{itemize}
  \item 1850 étudiants (1200 élèves-ingénieurs, 250 élèves-ingénieurs en formation continue et par apprentissage, 200 étudiants de Master, 200 doctorants),
  \item 550 chercheurs, enseignants-chercheurs et personnels de recherche,
  \item 150 personnels administratifs et techniques.
\end{itemize}
\paragraph{Le Lieu}
\begin{itemize}
  \item 40000 m\textsuperscript{2} de locaux,
  \item 1 campus de 16 ha,
  \item 5 laboratoires labellisés CNRS.
\end{itemize}
\paragraph{Centrale Nantes en France et dans le monde}
\begin{itemize}
  \item 11\textsuperscript{e} meilleure École d'Ingénieurs de France (classement 2010),
  \item 115 universités partenaires dans 40 pays.
\end{itemize}

\subsection{Les grandes têtes de l'école}\trad
Voici une petite liste (non exhaustive) de personnalités importantes du campus, que vous serez amenés à croiser durant vos 3 ans de thèse.
\todo{Compléter les noms et vérifier les nouveaux intitulés des fonctions}

\newcommand{\largcol}{0.29\textwidth}

\newcommand{\pers}[4]{%
\begin{tabular}{m{\largcol}}%
\ifthenelse{\equal{#1}{}}{\todo{Photo}}{}%
%\ifthenelse{\equal{#1}{}}{\todo{Photo}}{\includegraphics[width=2.5cm]{#1}}%
%\includegraphics{#1}
\\ \textbf{#2} \\ \textit{(#3)} \\ #4
\end{tabular}}

\noindent
\begin{tabular}{*{3}{p{\largcol}}}
  \pers{}{}{Directeur de l'école}{C'est notre chef à tous. On lui doit allégeance, même si en pratique, on a rarement affaire à lui.}&
  \pers{}{}{Directeur des études}{Comme son statut l'indique, il est davantage en lien avec les élèves. Amis moniteurs, si vous avez des problèmes avec vos enseignements, c'est peut-être à sa porte qu'il faut aller frapper.}&
  \pers{}{}{Directeur de la recherche}{C'est notre boss à nous. Il gère la politique de recherche de l'école (les projets, l'argent, les diverses activités, \dots) et peut s'occuper de nous en cas de [gros] problèmes avec notre thèse.}\\
  \pers{}{}{Secrétaire général}{Il s'occupe des grandes décisions administratives qui nous concernent (oui, la paperasse de l'inscription, c'est en partie pour son service\dots). Véritable encyclopédie des textes officiels, il pourra répondre aux questions auxquelles tout autre secrétaire aura jeté l'éponge.}&
  \pers{}{}{Directeur de la communication}{Lui et son équipe gèrent la diffusion des informations et l'événementiel sur le campus. Son travail dépend notamment de votre participation, alors n'hésitez pas à l'informer de tout ce qu'il se passe dans votre labo.}&
  \pers{}{}{Directeur adjoint}{C'est l'homme de terrain. Il met à exécution les grandes décisions « concrètes » de l'administration. Vous recevrez notamment pas mal de mails durant l'année sur d'éventuels travaux, changement d'organisation, etc. de sa part.}
\end{tabular}

\subsection{Les étudiants}\trad
L'école renferme une population d'étudiants assez hétéroclite.
Il y a :
\paragraph{Les élèves-ingénieurs (EI1, EI2 et EI3)} Ils sont admis à l'école sur concours après une classe prépa (bac+2) et constituent la majorité des étudiants du campus et sont présents pour 3 ans. Très actifs sur le plan associatif, ils animeront le campus tout au long de l'année. Les premières années vivent dans la résidence près du campus. Depuis peu, certains élèves ingénieurs travaillent en alternance au même titre que les ITII (voir plus bas).
\paragraph{Les Masters (M1 et M2)} Originaires d'un parcours universitaire tiers, les masters partagent souvent leurs cours avec les élèves ingénieurs. On peut notamment citer les Master Erasmus-Mundus attirant des étudiants du monde entier!
\paragraph{Les ingénieurs par alternance ITII Pays de la Loire (1ère, 2ème et 3ème année)} Il s'agit d'étudiants en formation par alternance (2 semaines en cours, 2 semaines en entreprise). L'ITII Pays de la Loire est un organisme de formation extérieur à l'École Centrale Nantes, cependant les cours se font sur le campus. Officiellement, ce ne sont pas des étudiants. Dans la pratique, ils sont exactement comme les autres, à la différence qu'ils ne sont présents que 2 semaines sur 4.
\paragraph{Les doctorants (D1, D2 et D3)} C'est toi.

\subsection{Les labos de l'école}\trad
Le campus de l'École Centrale comprend 5 labos (voir le plan du campus) :
\paragraph{Le GeM} De son vrai nom « Institut de Recherche en Génie civil et Mécanique ». Le Gem est un labo installé sur 3 sites : l'École Centrale, la fac de science et l'IUT de St Nazaire. On y fait de la simulation mécanique, du génie civil, des études de matériaux, de la dynamique rapide (crash-test)\dots Parmi tous les thèmes de recherche (et il y en a beaucoup), deux ressortent : La méthode X-FEM (méthode des éléments finis enrichis) et la PGD (méthode de décomposition de variables pour résoudre des problèmes beaucoup plus rapidement). D'autre part, un domaine d'application particulièrement en vogue ces temps ci est celui des matériaux composites.
\paragraph{Le LHEEA} (« Laboratoire de recherche en Hydrodynamique, Énergétique et Environnement Atmosphérique ») C'est sans doute le laboratoire qui possède le matériel expérimental le plus impressionnant du campus : un bassin de carène (pour tracter des maquettes de bateaux), un bassin de houle (pour faire des vagues), une soufflerie\dots Le LHEEA est morcelé en plusieurs équipes dont les principaux thèmes sont l'hydrodynamique et génie océanique (projet « SEAREV » de récupération d'énergie des vagues), l'énergétique des moteurs à combustion interne, l'étude de la dynamique de l'atmosphère en milieu urbain.
\paragraph{L'IRCCyN} (« Institut de Recherche en Communications et Cybernétique de Nantes ») Principalement installé dans le bâtiment S (le bâtiment relativement moderne, de forme ovale), l'IRCCyN concentre douze thématiques de recherche très variées. On y travaille par exemple sur la commande, la robotique, la conception, la fabrication additive, l'optimisation multidisciplinaire, les systèmes temps réels et même\dots la psychologie ! L'IRCCyN est réparti sur 4 sites de Nantes : l'École Centrale, l'École des Mines, l'École Polytechnique et l'IUT.
\paragraph{Le CERMA} (Centre de Recherche Méthodologique d'Architecture). Chapeauté depuis peu par l'École Centrale, le CERMA est avant tout un laboratoire dépendant de l'École Nationale Supérieure d'Architecture de Nantes. On y travaille notamment sur toute la physique liée l'architecture et l'environnement urbain.
\paragraph{Le laboratoire de mathématique Jean Leray} On n'en entend pas trop parler, mais il est là, caché dans une partie du bâtiment E ! En fait, c'est parce que l'essentiel se trouve à la fac de sciences et seul un petit bout est à l'École Centrale Nantes. La recherche à Centrale se concentre sur l'étude des équations aux dérivées partielles non-linéaire avec des domaines d'application pouvant aller des problèmes dynamique de populations à l'étude des changement d'état de matériaux supraconducteurs.

\subsection{Plan du site de l’École Centrale Nantes}\trad
\todo{Inclure le plan}

\section{Sport and culture on campus}
\subsection {Places to practice sports}
Do you wanna romp? Go out of your 10 m2 office?  Let's practice sport! Three structures are at your disposal around the campus.

\subsubsection{The Athletic Association Centrale Nantes (AS)}
It is the association managed by the sport association of th engineering students (BDS) from Centrale. Even if it is mainly dedicated to engineering students, it is commonly open to master and phd students. A membership fee can be asked to participate at some activities.
\paragraph{Contacts}
\begin{itemize}
  \item Email address of the sport association: \texttt{bds@ec-nantes.fr}
  \item Office: 1\textsuperscript{st} floor of L Building
  \item \texttt{http://bds.campus.ec-nantes.fr}
\end {itemize}
\paragraph{How to apply}
\begin{itemize}
  \item A cheque \EUR{55}, for the membership fee
  \item A medical certificate \footnote{-filled meadows certificates are available at the office of BDS} non-cons-indication (if you specify the competition)
\end{itemize}

\paragraph{Advantage}
\begin{itemize}
  \item [$+$] Many sports (17)
  \item [$+$] Possibility of academic competitions,
  \item [$+$] Diversity of levels,
  \item [$+$] Some offer sports training with teacher
  \item [$+$] You may find yourself playing with your own students!
\end{itemize}
\paragraph{Drawback}
\begin{itemize}
  \item [$-$] You may find yourself playing with your own students,
  \item [$-$] Sometimes too crowded,
  \item [$-$] Requires consistency in training.
\end{itemize}

\paragraph{Warning} The competitions are usually held on Thursday afternoon (often at the same time as seminars).


\paragraph{Some events:}
\begin{itemize}
  \item Outside Centrale: The Inter-Centrales, Challenge Lyon, BER (Brest), CAS (Supélec Rennes), the TOSS (Supélec Paris).
  \item In Centrale:  4 balls Tournament (T4B) 3 rackets Tournament (T3R), Inter-groups.
\end{itemize}

\subsubsection {Association Sportive du Personnel (ASP)}
It is the staff association of the Ecole Centrale (teachers, administrators, technicians, ... and PhD!). The association currently offers only four sports (futsal, tennis, gym and badminton) but is open to any contribution to develop other sports.

\paragraph{} during lunch time : 12:00 - 13:30.
\paragraph{Contacts}
\begin {itemize}
  \item Chairperson: \texttt{jean-paul.bouganne@ec-nantes.fr} (02 40 37 25 92)
  \item Intranet site of ECN
\end {itemize}
\paragraph{How to apply}
\begin{itemize}
  \item A cheque \EUR {13} (+ \EUR{13} for tennis \footnote {The tennis requires to pay a license FFT}).
  \item A medical certificate of non-cons-indication.
\end{itemize}

\paragraph{Advantage}
\begin{itemize}
  \item Cheap,
  \item Often fewer people,
  \item During lunch time.
\end{itemize}
\paragraph{Drawback}
\begin{itemize}
  \item During lunch time,
  \item Few sports (4),
  \item No competition possible (except for tennis!)
\end{itemize}

\subsubsection {The SUAPS}
SUAPS is the university sport association. It is located in the gymnase in front of the university restaurant "Le Tertre".
Even if you are not a student of the university, you can still sign up at SUAPS.
This is usually a good alternative if the sport you want to practice is not available in Centrale.

\paragraph{Warning} If, as a PhD student, you are not part of the university, you will be considered as "external student."
Registration will cost more (\EUR{50} instead of \EUR{35}), and you may have not access to some popular sports.
Visit the website for more information on SUAPS.

\paragraph{Advantage}
\begin{itemize}
  \item Many sports (47), including rare ones
  \item Lot of equipments,
  \item From beginners to experienced.
\end{itemize}

\paragraph{Weaknesses}
\begin{itemize}
  \item Sometimes too crowded,
  \item Off Centrale campus,
  \item You can be considered as "external student" in the university, and therefore not a priority in some sports.
\end{itemize}

\subsection {Cultural Associations Student}
\subsubsection {At the Ecole Centrale}
\paragraph{AED} Association of PhD Students from the Ecole Centrale Nantes. Its purpose is to connect PhD students from the differents laboratories located on the Centrale campus. Thus it organizes events, professional, cultural or even festive ones.
You can join this association on the Facebook group "ECN-AED" to keep up with the news. Website: \url{http://website.ec-nantes.fr/aed/}
\paragraph{AMAP Ecole Centrale} BDS and Central Green ECN offer local and organic fruits and vegetables, delivered each week. For more information, contact BDS.



%\section{Sport et culture sur le campus}\trad
%\todo{Vérifier les infos de cette section}
%\subsection{Où faire du sport ?}\trad
%Envie de te défouler ? De sortir de tes 10 m\textsuperscript{2} de bureau ?
%Va faire du sport ! Trois structures te permettent cela.
%
%\subsubsection{L'Association sportive de Centrale Nantes (AS)}
%C'est l'association gérée par le Bureau Des Sports (BDS) des étudiants de Centrale. En tant qu'étudiants, les doctorants peuvent s'inscrire à l'AS.
%\paragraph{Contacts}
%\begin{itemize}
%  \item L'adresse courriel du bureau des sports : \texttt{bds@ec-nantes.fr}
%  \item Bureau au 1\textsuperscript{er} étage du bâtiment L
%  \item \texttt{http://bds.campus.ec-nantes.fr}
%\end{itemize}
%\paragraph{Modalités d'inscription}
%\begin{itemize}
%  \item Un chèque de \EUR{55},
%  \item Un certificat médical\footnote{Des certificats prés-remplis sont disponibles au bureau du BDS} de non-contre-indication (précisez si vous faites de la compétition)
%\end{itemize}
%
%\paragraph{Points forts}
%\begin{itemize}
%  \item[$+$] Beaucoup de sports (17),
%  \item[$+$] Possibilité de faire des compétitions universitaires,
%  \item[$+$] Diversité de niveaux,
%  \item[$+$] Certains sports proposent des entrainements avec professeur,
%  \item[$+$] Tu peux te retrouver à jouer avec tes propres élèves !
%\end{itemize}
%\paragraph{Points faibles}
%\begin{itemize}
%  \item[$-$] Tu peux te retrouver à jouer avec tes propres élèves,
%  \item[$-$] Parfois trop de monde,
%  \item[$-$] Nécessite une régularité dans les entraînements.
%\end{itemize}
%
%\paragraph{Attention} Les compétitions ont généralement lieu les jeudis après-midi (souvent en même temps que des séminaires).
%Tout comme les entraînements, elles demandent un engagement particulier vis-à-vis vos équipes et de vos adversaires.
%Avant de t'engager, vérifie bien que tes obligations de doctorants te permettent de participer régulièrement à ces compétitions.
%
%\paragraph{Quelques événements tout au long de l'année}
%\begin{itemize}
%  \item Hors de Centrale : Les Inter-Centrales, le Challenge Lyon, le TEB (Brest), les CAS (Supélec Rennes), le TOSS (Supélec Paris).
%  \item Dans Centrale : Tournoi des 4 ballons (T4B), tournoi des 3 raquettes (T3R), Inter-groupes.
%\end{itemize}
%
%\subsubsection{L'Association Sportive du Personnel (ASP)}
%C'est l'association du personnel de l'École Centrale (professeurs, administratifs, techniciens, etc. et doctorants !). L'association ne propose que quatre sports (foot en salle, Tennis, musculation et badminton) mais ne demande qu'à évoluer. Si tu es motivé pour monter un sport, n'hésite surtout pas ! Pour info, il y avait avant du ping-pong et de la danse.
%
%\paragraph{Créneaux horaires} 12h--13h30.
%\paragraph{Contacts}
%\begin{itemize}
%  \item Président : \texttt{jean-paul.bouganne@ec-nantes.fr} (02 40 37 25 92)
%  \item Intranet du site de l'ECN
%\end{itemize}
%\paragraph{Modalités d'inscription}
%\begin{itemize}
%  \item Un chèque de \EUR{13} (+ \EUR{13} pour le tennis\footnote{Le tennis nécessite de payer une licence FFT}).
%  \item Un certificat médical de non-contre-indication.
%\end{itemize}
%
%\paragraph{Points forts}
%\begin{itemize}
%  \item Pas cher,
%  \item Souvent moins de monde,
%  \item Idéal pour aller se défouler sans prise de tête
%\end{itemize}
%\paragraph{Points faibles}
%\begin{itemize}
%  \item Peu de sports (4),
%  \item Pas de compétition possible (sauf pour le tennis !)
%\end{itemize}
%
%\subsubsection{Le SUAPS}
%Le SUAPS est l'organisme responsable des sports à la fac de Nantes.
%C'est une GROSSE structure basée dans le gymnase en face du resto U.
%Même si tu n'es pas étudiant de la fac, tu peux quand même t'inscrire au SUAPS.
%C'est en général une bonne alternative si le sport que tu désires n'est pas disponible sur le campus de Centrale.
%
%\paragraph{Attention} Si, en tant que doctorant, tu n'es pas inscrit à l'université, tu seras considéré comme « étudiant extérieur ».
%L'inscription te coûtera plus cher (\EUR{50} au lieu de \EUR{35}), et certains sports « populaires » ne te seront pas autorisés.
%Rendez-vous sur le site du SUAPS pour plus d'informations.
%
%\paragraph{Points forts}
%\begin{itemize}
%  \item Beaucoup de sports (47), dont des sports « rares »,
%  \item Beaucoup de matériel,
%  \item Grande diversité de niveaux,
%  \item Propose des créneaux pour la piscine.
%\end{itemize}
%\paragraph{Points faibles}
%\begin{itemize}
%  \item Parfois trop de monde,
%  \item Hors du campus,
%  \item Tu peux être considéré comme « étudiant extérieur » à la fac, et donc non-prioritaire sur certains sports.
%\end{itemize}
%
%\subsection{Associations culturelles étudiantes}\trad
%\subsubsection{À l'École Centrale}
%\paragraph{AED} Association des Élèves en Doctorat de l'École Centrale Nantes. Elle fédère tous les doctorants présents sur le site de l'ECN et organise des événements tant festifs que scientifiques. Rejoins-nous sur le groupe Facebook « AED-ECN » pour voir le fil des infos. Site internet : \url{http://website.ec-nantes.fr/aed/}
%\paragraph{AMAP Ecole Centrale} Le BDS et Centrale Vert de l'ECN propose à la rentrée 2011 des paniers de légumes bio, distribués toutes les semaines. Tu auras de plus amples informations à la rentrée 2011-2012, ou par mail.
%
%D'autre part, les bureaux des élèves (BDE, BDA, BDS) organisent en Septembre une soirée de rentrée des associations. Tu pourras y découvrir tous les clubs et éventuellement t'y inscrire ! Renseigne-toi sur la date.
%
%\subsubsection{Sur le campus universitaire}
%\paragraph{Association « Des pieds dans la tête »} Propose des ciné-philo ; le but de l'association est de sortir la philosophie du cadre de l'enseignement et d'ouvrir des débats plus larges par exemple sur la liberté. Site : \url{http://depidalate.free.fr/}
%\paragraph{Maison des Jeux} Tous les mercredis midi au TU, tu pourras découvrir et participer à des jeux de société et jeux en bois.



%\section{Trouver un logement}\trad
\section{Find a home}
%\paragraph{Cité universitaires} En tant que doctorant tu es normalement prioritaire pour obtenir un logement du CROUS. Le mieux est de se rendre directement  à l'accueil du CROUS (2 bd Guy Mollet arrêt de TRAM Faculté) et se référer au site internet du CROUS www.crous-nantes.fr. ATTENTION, il y a beaucoup de demandes, il est donc préférable de s'y prendre plusieurs mois à l'avance (au mois d'Avril pour Septembre) !
\paragraph{Campus}
As a PhD student you are normally a priority for housing CROUS.
It is best to go directly to the home CROUS (2 bd Guy Mollet stop TRAM Faculty) and refer to CROUS website www.crous-nantes.fr.
ATTENTION, there are many applications, it is preferable to take several months in advance (in the month of April to September)!

%\paragraph{Colocation} Si vivre à plusieurs te tente, tu peux trouver sur www.appartager.fr plein d'annonces gratuites pour partager les frais et faire des rencontres très rapidement.

\paragraph{Flatshare}
If you live several attempts, you can find lots of ads on www.appartager.fr free to share costs and meet new people very quickly.

%\paragraph{Location d'appartement} Pour trouver un appartement à louer, il y a deux possibilités : les agences immobilières (ATTENTION, il faut généralement payer des frais d'agence pouvant s'élever à 1 mois de loyer), les annonces de particuliers (leboncoin.fr, de particulier à particulier pap.fr et/ou le CROUS). Penses également à acheter le supplément immobilier du journal Ouest France publié tous les samedi !

\paragraph{Apartment rental}
To find an apartment, there are two possibilities: the estate (WARNING must generally pay agency fees of up to one month's rent), announcements of special (leboncoin.fr of particular particular pap.fr and / or CROUS).
Do also buy real estate supplement of the newspaper Ouest France published every Saturday!

%Pour le lieu de ton logement tu feras face au dilemme classique :
%\begin{itemize}
%  \item Habiter au centre ville, près de toutes les animations étudiantes, facilement accessible par le tram pour te rendre au boulot. Mais tu risques d'avoir un petit logement cher avec des difficultés pour garer ta voiture.
%  \item Habiter une petite maison pas très chère en banlieue au milieu de la nature avec jardin et barbecue mais tu devras investir dans une voiture et faire une croix sur le troisième verre pour rentrer chez toi...
%  \item Dormir dans ton bureau (tu remarqueras comme la moquette est confortable... et propre !!)
%\end{itemize}

For the place of your accommodation you will face the classic dilemma:
\begin{itemize}
  \item Living in the city center, close to all the students animations, easily accessible by tram you get to work. But you have a small risk of expensive housing with difficulties to park your car.
  \item Living in a small house in the suburbs not very expensive in the middle of nature with garden furniture and barbecue, but you will have to invest in a car and make a cross on the third glass to go home \dots
  \item Sleep in your office (as you will notice the carpet is clean and comfortable \dots!)
\end{itemize}

%Lors de ta première année de thèse, tu peux avoir droit aux APL (Aides Personnalisées au Logement) puisqu'elles se basent sur les revenus que tu as obtenus l'année précédente pour calculer ton aide (penses à valoriser ton statut d'étudiant). Se renseigner sur le site \url{www.caf.fr} de la Caisse d'allocation familiale qui délivre cette aide personnalisée. En tout cas, mieux vaut faire la demande, ça ne mange pas de pain.
During your first year PhD student, you may be eligible for APL (Custom Aids for Housing) as they are based on the income you have from the previous year to calculate your help (thinking about enhancing your student status).
Learn about the site \url{www.caf.fr} the Family Allowance Fund which delivers the personalized assistance.
In any case, it is better to apply, it does not eat bread.

%\section{Manger près de Centrale}\trad
\section{Eat near Central}

%\subsection{Les restaurants universitaires (ou « RU »)}
\subsection{University restaurants (or "RU")}
%Les « restos U » auxquels tu peux avoir facilement accès à pied depuis Centrale se comptent au nombre de trois : le Tertre, le Rubis et la Cafétéria du pôle étudiant. Il en existe d'autres à Nantes et tu trouveras toutes les informations les concernant sur le site du CROUS de Nantes (\url{http://www.crous-nantes.fr}). Le tarif d'un plateau-repas au RU se situe aux alentours des 3 euros (tarif 2011 – 2012).
%Le paiement s'effectue exclusivement grâce à ta carte d'étudiant Pass'Sup, qui fait aussi office de carte Moneo.

%Tu peux créditer ta carte Pass' Sup dans n'importe quels bureaux de poste, banques, cabines téléphoniques, halls d'accueil de restaux U, etc... arborant le logo « Moneo », ou encore en ligne sur \url{www.moneo.net} !!

The "restos U" that you can easily access on foot from Central to count the number three: Tertre, the Ruby and the Cafeteria division student. There are others to Nantes and you will find all the information about them on the site CROUS Nantes (\url{http://www.crous-nantes.fr}). The price of a meal tray in the UK is around 3 euros (2011 - 2012).

Payment is made only through your student card Pass'Sup, which also serves as Moneo card.

You can credit your card Pass' Sup in any post office, banks, phone booths, reception halls of restaux U, etc \dots bearing the logo "Moneo" or online at \url{www.moneo.net}!

%\subsubsection{Le Tertre}
\subsubsection{The Tertre}
%Le Tertre se situe juste en face de l'arrêt de tramway Facultés. Il comporte deux étages et permet un choix relativement varié de cuisines.
%On y trouve au rez-de-chaussée une sandwicherie ainsi qu'un coin café et à l'étage plusieurs pôles thématiques (cuisine du monde, poisson, grill, pizza, etc...).

%Ouverture : Le midi (11h00-13h30) du lundi au vendredi.

The Tertre is located just opposite the tram stop Faculties. It has two floors and provides a relatively varied choice of cuisines.
It is on the ground floor a sandwich and a coffee corner and upstairs several thematic clusters (world cuisine, fish, grill, pizza, etc \dots).

Opening: afternoon (11:00 to 1:30 p.m.) Monday through Friday.
%\subsubsection{Le Rubis}
\subsubsection{The Rubis}

%L'intérêt principal de ce restaurant est qu'il est en général moins pris d'assaut que son homologue le Tertre et qu'en plus il est ouvert le soir (avec une sympathique petite musique de fond bien agréable pour digérer toutes les formules et équations de la journée).
%Ainsi, tu pourras travailler ta thèse jusqu'à 19h00, manger au RU et enfin revenir pour poursuivre ton travail (jusqu'à la fermeture de l'école à 22h00 pile)

%Ouverture : Le midi (11h00-13h30) et le soir (18h30-20h00) du lundi au samedi midi.

The main interest of this restaurant is that it is generally less attacked than his counterpart Tertre and in addition it is open in the evening (with a nice little nice background music to digest all formulas and equations of the day).
Thus, you can work on your thesis until 19:00, eating in the UK and finally back to continue your work (up to the closure of the school at 22:00 battery)

Open: For lunch (11h00-13h30) and evening (18h30-20h00) from Monday to Saturday.


%\subsection{La Cafétéria du pôle étudiant}
\subsection {The Student Cafeteria pole}
%La cafétéria du pôle étudiant se situe entre les Facs de Droit et de Lettres.
%Elle est ouverte en continu de 8h30 à 17h30 et l'ambiance est assez "chaleureuse" (ambiance Lettre et Droit quoi...).
%On y sert des repas "chauds" en continu (paninis, tapas, etc...) et tu y trouveras aussi crudités et viennoiseries.

%Ouverture : Du lundi au Vendredi de 9h30 à 21h00.

The student center cafeteria is located between Facs Law and Letters.
It is open continuously from 8:30 to 17:30 and the atmosphere is very "warm" (Letter atmosphere and what \dots Right).
It serves meals "hot" continuously (paninis, tapas, etc \dots) and you'll also find salads and pastries.

Hours: Monday to Friday from 9:30 to 21:00.

%\subsection{Le bâtiment L (la K-fet' de l'école)}

%Il fut un temps pas si lointain où le bâtiment L de l'école abritait une cafétéria permettant aux élèves et au personnel de s'y restaurer sans quitter le campus.
%Cette K-fet' ayant fermé ses portes pour des raisons administratives en 2012, il n'est plus possible de s'y rendre le midi.
%Cependant, la rumeur court qu'elle pourrait renaître de ses cendres ; affaire à suivre...
%Pas traduit car pas intéressant

%\subsection{La baraque à frites}
%Quand les RU sont fermés ou quand le temps te manque, tu trouveras en général une camionnette-snack garée au niveau du rond point de l'école.
%Double-américain avec surplus de ketchup et frites au menu !!

%Ouverture : Le midi du lundi au vendredi.

\subsection{The French Fries}
When the RU are closed or when you lack the time, you will find usually a van parked snack at the roundabout school.
Double-American with extra ketchup and fries on the menu!

Opening: afternoon of Monday to Friday.

%\subsection{Le Fac' Food}

%Le Fac' Food est, tu l'as deviné, une sandwicherie. Elle est située en face de l'école et tu y trouveras sandwichs, burgers en tous genres, paninis, américains, etc...

%Ouverture : Du lundi au vendredi de 8h00 à 15h00.

\subsection{The Fac 'Food}

The Fac 'Food is, you guessed it, a sandwich shop. It is located in front of the school and you will find sandwiches, burgers of all kinds, paninis, American, etc \dots

Hours: Monday to Friday 8:00 to 15:00.

%\subsection{Pizzeria-brasserie « Le Café Gourmand »}

%Cette brasserie jouit d'une bonne réputation sur le net et elle est pourvue d'une grande terrasse-jardin à l'arrière.
%Formule "Pizza - Café Gourmand" à 14 euros.

%Ouverture : Le midi du lundi au vendredi et le jeudi soir (pizza à emporter)

\subsection{Pizzeria-brewery "Café Gourmand"}

This beer has a good reputation on the net and it is provided with a large garden terrace at the rear.
Formula "Pizza - Gourmet Coffee" to 14 euros.

Opening: afternoon of Monday to Friday and on Thursday evening (pizza takeaway)

%\subsection{Bar-brasserie « Les Facultés »}
%La brasserie Les Facultés est située juste en face de l'école à côté du Fac' Food (même enseigne).
%Tu pourras y manger en terrasse et tu y trouveras également un bureau de tabac.

%Ouverture :     Du lundi au vendredi de 7h00 à 20h00 et le samedi de 7h30 à 14h00.

\subsection {Beer Hall "Faculties"}
The brewery is located Faculties in front of the school next to the Fac \'Food (same brand).
You can eat on the terrace and you will find also a tobacconist.

Hours: Monday to Friday from 7:00 to 20:00 and Saturday from 7:30 to 14:00.

%\subsection{Bar-brasserie « Le Campus »}
%Cette petite brasserie avec terrasse située tout près de Centrale, rue du Fresche Blanc, t'accueillera avec des formules midi de 7 à 15 euros (+ buffet à volonté).

%Ouverture : Du lundi au vendredi de 7h00 à 17h00.

\subsection {Beer Hall "Campus"}
This little brasserie with terrace located close to Central Street Fresche White will welcome you with formulas noon 7 to 15 euros (+ buffet).

Hours: Monday to Friday 7:00 to 17:00.

%\section{Aide aux étudiants étrangers}\trad
%Tu es un chercheur étranger à Nantes ?
%L'association « Chercheurs Étrangers à Nantes » constitue un guichet unique en matière d'information et d'accueil.
%Parmi les services proposés, on retrouve :
%\begin{itemize}
%  \item L'aide à la préparation du séjour,
%  \item Des services personnalisés pour ton arrivée (information sur la vie quotidienne, transport, plans, scolarisation des enfants...),
%  \item Un appui aux démarches administratives de séjour (constitution et dépôt du dossier de demande de titre de séjour, informations pratiques sur les taxes, salaires, assurances, visite médicale, banque, etc...).
%  \item Des cours de français en tant que langue étrangère (à faible coût), niveaux débutant et avancé
%\end{itemize}

\section{Help foreign students}
You are a foreign researcher in Nantes?
The association "Foreign Researchers in Nantes" is a one-stop information and care.
Among the services offered are:
\begin{itemize}
   \item Support for the preparation of stay
   \item personalized services for your arrival (information on daily life, transport, maps, children's schooling \dots)
   \item Support to administrative room (Incorporation and submission of the application for a stay, practical information on taxes, wages, insurance, medical, banking, etc \dots).
   \item French courses as a foreign language (at low cost), beginner and advanced levels
\end{itemize}
%Par ailleurs, tu as envies de participer, de créer ou de t'engager ?
%Les associations à Nantes sont à ton écoute !
%Tu trouveras des associations impliquées dans plusieurs secteurs d'activité.
%N'hésite pas à visiter le site internet de l'Université de Nantes (rubrique vie associative) pour plus d'informations.

In addition, you desire to participate, create or get involved?
Associations Nantes are your listening!
You will find associations involved in several industries.
Do not hesitate to visit the website of the University of Nantes (item associations) for more information.

%\subsection{Association de regroupement culturel}
%Les associations des étudiants étrangers ont pour but d'être un repère pour un étudiant étranger à Nantes et de promouvoir leurs cultures (danse, alimentation, musique, ...).
%On retrouve l'association des étudiants libanais à Nantes \textbf{ANADYL} (\texttt{asso\_anadyl@yahoo.fr}), l'association des étudiants guinéens à Nantes \textbf{AGEN} (\texttt{agen.nantes@yahoo.com}), l'association des étudiants vietnamiens à Nantes \textbf{AEVN} (\texttt{contact@aevn.fr}), l'association des étudiants gabonais à Nantes \textbf{GaboNantes} (\texttt{gabo\_nantes2@yahoo.fr}), l'association des étudiants italiens à Nantes \textbf{Issimo.it} (\texttt{issimo.nantes@gmail.com}) et l'association des étudiants Roumains à Nantes \textbf{Roumanie d'ici} (\texttt{roumaniedici@yahoo.fr}).
%N'hésite pas à te faire connaître auprès de ces associations.
%Et d'ailleurs si tu es Français n'hésite pas non plus à te faire passer pour un ERASMUS car en France, il n'y a pas d'associations d'étudiants français !

\subsection{Cultural Association grouping}
Associations of students intended to be a benchmark for foreign students in Nantes and promote their culture (dance, food, music, \dots).
\begin{description}
    \item[ANADYL] the association of Lebanese students in Nantes \texttt{associated\_anadyl@yahoo.fr})
    \item[AGEN] the Association of Guinean students in Nantes (\texttt{agen.nantes@yahoo.com})
    \item[AEVN]  the Association of Vietnamese students in Nantes (\texttt{contact@aevn.fr)}
    \item[GaboNantes] the Association of Gabonese students in Nantes (\texttt{gabo\_nantes2@yahoo.fr})
    \item[Issimo.it] the association of Italian students in Nantes (\texttt{issimo.nantes@gmail.com})
    \item[Roumanie d'ici] the Association of Romanian students in Nantes (\texttt{roumaniedici@yahoo.fr}).
\end{description}
Do not hesitate to make you know with these associations.
And besides, if you're French does not hesitate to make you spend more for an ERASMUS because in France, there are no associations of French students!


\chapter{Visiter Nantes}\trad

Capitale de la région des Pays de la Loire, Nantes est aujourd'hui la 6ème ville de France et se situe au centre d'une agglomération de plus de 580 000 habitants ! Mais, pas de panique, 35 \% de la population a moins de 25 ans ! Ça en fait du monde à rencontrer ! Et oui, la douceur de vivre à Nantes a attiré un afflux record de citadins et continue de séduire bien des touristes.
Dans cette partie, vous retrouverez plusieurs plans de la ville et des environs, avec pleines de bonnes idées pour découvrir et s'amuser !

\section{Pour bien manger}\trad

\paragraph{Les restaurants} le coin resto/bars le plus fréquenté est le quartier Bouffay (situé entre les repères 2 et 9). Tu y trouveras une foultitude de crêperies, brasseries, japonais et pubs.
\paragraph{Faire ses courses} A Nantes, n'essaie pas de chercher un Cora ou un Simply, tu devras principalement choisir entre :
Leclerc (au Centre commercial Atlantis à Saint Herblain, entre autres)
Carrefour (au centre commercial de Beaulieu) ou les petits carrefour contact et carrefour city
Auchan (route de Vannes à Saint Herblain)
Monoprix
\paragraph{Faire son marché} Pour celles/ceux qui aiment les légumes/fruits de saison, un marché se tient
tous les samedis matin (jusqu'à 13h, heureusement pour les lève-tards) place Gloriette, en face de l'arrêt Médiathèque (M1).
tous les matins de la semaine à Talensac
un petit marché bio le mercredi matin de 8h à 13h à Commerce, sur le square derrière la Fnac.

\section{Point Sport et culture}\trad
\paragraph{Piscine}
A Nantes, 6 piscines sont disponibles. Tu trouveras dans le centre la piscine Léo Lagrange équipée d'un bassin sportif de 50 mètres. A 3 minutes de Centrale, la piscine du Petit Port toute neuve te permettra de te défouler quand tu pètes un plomb sur ton code Matlab ! L'entrée est d'environ 3 euros mais il existe également des cartes à points qui permettent de bénéficier de tarifs préférentiels et des réductions étudiantes. Plus d'infos sur le site de la ville : www.nantes.fr/detente/piscines-nantaises/
\paragraph{Parcs}
Le parc de Procé       est le parc le plus apprécié en ville ; on peut le regagner à pied en suivant une rivière que l'on pourra rejoindre à partir de la rue de Lamoricière (proche arrêt Chantier Naval). Le parc est également idéal pour ceux qui pratiquent le footing ; en effet, après avoir traversé le parc, on peut passer des portes grillagées qui mènent à une zone plus champêtre (où se situe la piscine des Dervallières, un terrain de foot et des terrains de tennis). Au final, on peut faire un parcours en boucle d'environ 10 km à partir de l'entrée du parc.
A la belle saison, tu peux aussi profiter de la fraicheur du Jardin des Plantes        ; la collection impressionnante de fleurs et l'aménagement paysager avec ses fontaines t'en mettent plein la vue ! Idéal pour venir bouquiner tranquille également... ou tout simplement attendre ton train.
\paragraph{Cinémas}
Sur la place du commerce, à côté de la Fnac, se tient un cinéma Gaumont Pathé. Pratique lorsqu'on habite au centre-ville et que l'on veut enchaîner sur le restaurant d'à côté, mais les prix restent élevés (7-8 euros en étudiant) !
Au centre commercial Atlantis, à l'ouest de Nantes (accès : ligne 1 ou voiture), il existe deux complexes, Pathé Atlantis et UGC Atlantis. Ils proposent des places en -26 ans aux alentours de 4 euros pour toutes les séances.

\section{Où sortir à Nantes ?}\trad
Pour la fête, on va te faire confiance. Sinon, voici trois coups de pouce :
\begin{itemize}
  \item Le magazine « Pulsomatic » recense tous les spectacles et concerts de Nantes et sa région ; tu peux trouver la version papier à l'accueil des RU, du CROUS ou d'autres établissements étudiants, sinon rend-toi directement sur le site \url{www.pulsomatic.com}.
  \item Le site www.onvasortir.com te met en relation avec des personnes de ta ville/région pour rencontrer et s'amuser autour de thèmes communs.
  \item Va profiter des transats, boire un verre ou danser au hangar à bananes : le long du quai des Antilles, les anciens entrepôts où murissaient les bananes importées de Guinée ou de Guadeloupe ont été aménagés pour accueillir galeries d'art, restaurants, une boite de nuit (le LC) et des bars à thèmes. Le long du quai, tu peux également voir les 18 anneaux de Buren, que tu apprécieras plus à la nuit tombée, lorsqu'ils s'éclairent. Et la grande grue jaune à la pointe de l'île s'appelle Titan !
\end{itemize}

\section{A faire absolument !}\trad
\begin{itemize}
  \item Se balader dans le jardin japonais de l'île de Versailles au cadre très romantique (attention, il y a du couple au m\textsuperscript{2}...). Pour les sportifs, on peut également louer des canoës-kayak. Durant le mois de Juin, tu iras écouter des concerts de jazz gratuits dans le cadre du festival « Les Rendez-Vous de l'Erdre ».
  \item Faire le tour des remparts du château des ducs de Bretagne, c'est gratuit et tu auras un beau panorama : attention, ça vente ! Le musée du château est aussi très sympa. L'histoire de Nantes vous y sera contée avec les derniers outils intéractifs du moment. Idéal pour s'occuper à Nantes un jour de pluie (tellement rare dans cette région..) ou encore pour trouver un coin au frais les jours de canicule.
  \item Prendre le navibus pour Trentemoult (entre les arrêts Chantier Naval et Gare maritime, de la ligne de métro 1) ; ancien village de pêcheur, Trentemoult est maintenant un quartier branché, car bien restauré par les nouveaux habitants. Il donne une agréable sensation de dépaysement, à quelques minutes seulement du centre-ville. Le navibus est considéré comme une navette classique du réseau tan, ce qui implique que le prix du ticket n'est pas plus cher que pour le reste du réseau (et que la carte d'abonnement suffit ! si vous l'avez). Ne cherches pas ailleurs, les seuls bars/restaurants sont ceux que tu vois en arrivant de la navette.
  \item Visiter la galerie des machines et regarder marcher l'Eléphant ; et oui, un éléphant vit sur l'île de Nantes et vous le trouverez sagement parqué dans son hangar à droite à la sortie du pont Anne de Bretagne. Il est le premier de toute une série de machines articulées et de projets fantasmagoriques, qui font parti du projet de réaménagement de l'île de Nantes. Dans la galerie des machines, des animateurs vous expliqueront la construction de l'Eléphant et d'autres machines et vous pourrez même en tester certaines ! Ça fait rêver... Pour connaître les horaires ou les tarifs, se rendre sur le site : \url{www.lesmachines-nantes.fr}
  \item Le hangar à bananes
  \item Se balader sur les berges de l'île de Nantes, observer les vieux gréements qui y séjournent ou qui sont juste de passage.
\end{itemize}

\section{Événements à ne pas louper !}\trad
\begin{itemize}
  \item Ne pas rater le passage de ROYAL DELUXE avec ses géants qui sillonnent le centre-ville
  \item Festival de musique classique : les « folles journées de Nantes » aux mois de janvier-février
  \item Les journées Escapades (un week-end entre avril et mai), durant lesquelles vous sont proposées des initiations à des sports de plein air et des démonstrations (les sports au programme : kite-surf, catamaran, tir-à-l'arc, speed-sail, équitation, char-à-voile, beach soccer, beach volley) ; retrouvez le programme sur le site de la région Loire-Atlantique ou dans le magazine gratuit de la région Loire-Atlantique que vous recevez normalement par courrier !
\end{itemize}

\section{Les plus :}\trad
\begin{itemize}
  \item Pense à l'abonnement de tram, remboursé en moitié par ton employeur (par l'école centrale si tu es embauché par celle-ci) !
  \item Pense à venir à l'école en vélo dès les beaux jours (en voiture, c'est la galère !). Des locations de vélos sont proposées aux étudiants pour \EUR{45} par an (Vélocampus). Et le réseau Bicloo s'étend jusqu'à l'Ecole Centrale. Les bords de l'Erdre pour aller au boulot c'est le pied !
  \item Et pour ceux qui veulent connaitre toutes les facettes de Nantes et aiment les jeux de pistes, vous trouverez votre bonheur avec les cistes : \url{http://www.cistes.net/}
\end{itemize}





\chapter{What will it happen to you in thesis?}

\section{Sign PhD}
It's a bit complicated but you'll get there.
You will have to moult back and forth between your desktop printer, the secretariat located in Building A, your supervisors, laboratory management and the graduate school to which you are attached.
Not to mention that little circle your internal laboratory between managers and secretary to the Director of the Laboratory for he gives you a desk, PC, email address \dots
One very important thing that you must read is the thesis charter is a contract between you, your manager and the director of the doctoral school.
It says basically that you must do your best in most conditions you must provide your manager, and all overseen by the director of the doctoral school, ensures the smooth running of your thesis.

\section{The thesis grants}
Let us begin by recalling the basics: What is a scholarship?
Well to perform a thesis, it is first necessary funding (this word is also sometimes used as a synonym for "Exchange").
This funding is used to pay for such testing campaigns, hardware or possibly as PhD student \dots
Normally, if you read this book for PhD at the moment is that you must already have a scholarship (in any case, hopefully for you) \dots
You must therefore tell you that the previous paragraph you do not learn much, but your purse is a purse among many others!
Indeed, there are many.
This section of the GSD has also not pretend to want list exhaustively all types of existing scholarship, but at least we will present the most common funding (for your general culture).
To put it simply, there are three kinds of funding:
\begin{enumerate}
  \item \textbf{public funding:} They come from state agencies or local governments. Include scholarships:
  \begin{itemize}
    \item the Ministry of Research (ANR grants: National Agency for Research, Scholarship MENRT, scholarships for ENS AN \dots)
    \item regions (or more generally public authorities)
    \item CNRS (Doctorate Scholarships for Engineers: BDI, \dots)
    \item other research organizations (INSERM, INRIA, INRA, INED, CNES, IFREMER, ONERA, ADEME, ANVAR \dots)
  \end{itemize}
  \item \textbf{Private financing:} In this case, this is a company that pays the PhD student to work on a particular subject in collaboration with a research laboratory. The most common are scholarships CIFRE (Convention Industrielle de Formation by Research). These are scholarships that have a strong link with the company, and that can lead to a permanent contract.
  \item \textbf{scholarships for foreign students:} Depending on the country, it is possible for some students under certain conditions to obtain a PhD scholarship from their government and / or the French government. If you are in this case, information yourself from your graduate school, some require laboratories to ensure you a minimum wage.
\end{enumerate}

\section{The vocabulary of the thesis}
For you the master of words conf ', postdoc, etc.. mingle happily in your head?
This section will help you then.
\paragraph {Postdoc}
As its name suggests, a postdoc is performed after a PhD.
It allows for a period of 1 to 3 years of rich professional experience.
A postdoc is often required to access the items described below.
\paragraph {Lecturer (TM)}
It is often the status of your supervisor.
It has both a higher teaching load of about 200 h coupled with research.
To become a master conf must be qualified by a committee of the CNU (National Council of Universities) in the given course and get a job \dots
\paragraph{Professor University (PU)}
Your supervisor surely belongs to this species.
Unlike the master conf ', has the right to supervise PhD students.
To become a PU, it is necessary to pass the Habilitation Research (HDR) and of course to get a job \dots
\paragraph{Research Fellow and Director of Research}
These are respectively equivalent MC and PU but the teaching load.
They are attached to research organizations such as CNRS, INRIA, INCERM, etc..
\paragraph{Associate Professor (PRAG)} This is a status corresponding to a load of about 400 hours of teaching (not research).
Associate? This term means that the individual is the holder of aggregation, that is to say it has been received in the national competition of the same name.
Aggregation allows for example to teach in preparatory class.
The competition (which includes a theoretical part and a learning part) can prepare one or an SLA that offers a year of preparation framed.

\section{Working after thesis}
What will you do after your PhD?
The first answer is going to be: change your identity papers for the acronym "Dr." as you expected!
But apart from that, what positions are open to you?
\begin{itemize}
  \item engineer / researcher in the industry. The industry is recruiting more doctors, especially in large companies. This hiring is favored by tax charges for the use of a doctor permanent contracts via the Research Tax Credit (CIR). For example, taking advantage of a CIR, the doctor would not cost a sub during the first 2 years.
  \item Teaching pure
  \item + Education Research
  \item Basic Research
\end{itemize}

For the last three options, refer to the definitions below.
Anyway, it is better to act earlier in the thesis based on career prospects.
For example, a doctoral student who wishes to become a researcher's interest to publish much during his PhD.
%\chapter{Que va-t-il t'arriver en thèse ?}\trad
%
%\section{S'inscrire en doctorat}\trad
%C'est un peu compliqué mais tu vas y arriver.
%Tu devras faire moult aller-retours entre ton bureau, l'imprimante, le secrétariat situé au bâtiment A, tes encadrants, la direction du laboratoire et l'école doctorale à laquelle tu es rattaché.
%Sans parler de ce petit cercle vicieux interne à ton laboratoire entre les responsables informatiques et secrétaire du directeur de Laboratoire pour qu'il te donne un bureau, un PC, une adresse mail\dots
%Une chose très importante que tu devras lire est la charte des thèses qui est un contrat passé entre toi, ton directeur et le directeur de l'école doctorale.
%Il précise en gros que tu devras faire ton maximum dans les meilleures conditions que doit t'offrir ton directeur, et le tout supervisé par le directeur de l'école doctorale, garant du bon déroulement de ta thèse.
%
%\section{Les bourses de thèse}\trad
%Commençons tout d'abord par rappeler les bases : qu'est-ce qu'une bourse ?
%Et bien pour pouvoir effectuer une thèse, il faut d'abord un financement (ce mot est d'ailleurs parfois utilisé comme synonyme de « bourse »).
%Ce financement sert par exemple à payer les campagnes d'essai, le matériel informatique ou éventuellement le thésard aussi\dots
%Normalement, si tu lis ce recueil pour doctorant en ce moment, c'est que tu dois déjà avoir une bourse (en tout cas, on l'espère pour toi)\dots
%Tu dois donc te dire que le précédent paragraphe ne t'a pas appris grand-chose, mais ta bourse, c'est une bourse parmi tant d'autres !
%En effet, il en existe une multitude.
%Ce paragraphe du GSD n'a d'ailleurs pas la prétention de vouloir répertorier exhaustivement tous les types de bourses existants, mais nous allons au moins te présenter les financements les plus courants (pour ta culture générale).
%Pour faire simple, il y a 3 sortes de financements :
%\begin{enumerate}
%  \item \textbf{Les financements publics :} Ils proviennent d'organismes de l'état ou des collectivités locales. On peut citer les bourses :
%  \begin{itemize}
%    \item du ministère de la recherche (bourses ANR : Agence Nationale pour la Recherche, bourses MENRT, bourses AN pour normaliens, \dots)
%    \item des régions (ou plus généralement des collectivités publiques)
%    \item du CNRS (Bourses de Doctorat pour Ingénieurs : BDI, \dots)
%    \item d'autres organismes de recherche (INSERM, INRIA, INRA, INED, CNES, IFREMER, ONERA, ADEME, ANVAR, \dots)
%  \end{itemize}
%  \item \textbf{Les financements privés :} Dans ce cas, c'est une entreprise qui paye le thésard pour travailler sur un sujet particulier en collaboration avec un laboratoire de recherche. Les plus répandues sont les bourses CIFRE (Convention Industrielle de Formation par la Recherche). Ce sont des bourses qui présentent un lien fort avec l'entreprise, et qui peuvent conduire à un CDI.
%  \item \textbf{Les bourses pour étudiants étrangers :} Selon les pays, il est possible à certains étudiants étrangers et sous certaines conditions d'obtenir une bourse de thèse auprès de leur gouvernement et/ou du gouvernement français. Si tu es dans ce cas, renseigne-toi auprès de ton école doctorale, certaines demandent aux laboratoires de t'assurer un salaire minimum.
%\end{enumerate}
%
%\section{Le vocabulaire de la thèse}\trad
%Pour toi les mots maître de conf', postdoc, etc. se mélangent allègrement dans ta tête ?
%Cette section va alors t'aider.
%\paragraph{Postdoctorant}
%Comme son nom l'indique, un postdoc est effectué après un doctorat.
%Il permet pendant une période de 1 à 3 ans d'enrichir son expérience professionnelle.
%Un postdoc est souvent requis pour accéder à aux postes décrits ci-après.
%\paragraph{Maître de conférences (MC)}
%C'est bien souvent le statut de ton encadrant.
%Il a à la fois une charge d'enseignement supérieur d'environ 200 h couplée à de la recherche.
%Pour devenir maître de conf, il faut être qualifié par un comité du CNU (Conseil National des Universités) dans la section considérée et bien sûr décrocher un poste\dots
%\paragraph{Professeur des universités (PU)}
%Ton directeur de thèse appartient sûrement à cette espèce.
%À la différence du maître de conf', lui a le droit de superviser des doctorants.
%Pour devenir PU, il est nécessaire de passer l'Habilitation à Diriger les Recherche (HDR) et aussi bien sûr de décrocher un poste\dots
%\paragraph{Chargé de recherche et directeur de recherche}
%Ce sont respectivement les équivalents de MC et PU mais sans la charge d'enseignement.
%Ils sont rattachés à des organismes de recherche tels que le CNRS, l'INRIA, INCERM, etc.
%\paragraph{Professeur agrégé (PRAG)} C'est un statut correspondant à une charge d'environ 400h d'enseignement (et pas de recherche).
%Agrégé ? Ce terme signifie que l'individu est titulaire de l'agrégation, c'est-à-dire qu'il a été reçu au concours national du même nom.
%L'agrégation permet par exemple d'enseigner en classe préparatoire.
%Le concours (qui comprend notamment une partie théorique et une partie pédagogique) peut se préparer seul ou à une ENS qui propose une année de préparation encadrée.
%
%\section{Travailler après la thèse}\trad
%Que vas-tu faire après ta thèse ?
%La première réponse qui va de soit est : changer tes papiers d'identités pour obtenir l'acronyme « Dr » que tu attendais tant !
%Mais à part ça, quels postes s'ouvrent à toi ?
%\begin{itemize}
%  \item Ingénieur/chercheur dans l'industrie. L'industrie recrute de plus en plus des docteurs surtout dans les grandes entreprises. Cette embauche est favorisée par des allègements de charges pour l'emploi d'un docteur en Contrat à Durée Indéterminée via le Crédit Impôt Recherche (CIR). A titre d'exemple, en profitant d'un CIR, le docteur ne couterait pas un sous durant les 2 premières années.
%  \item Enseignement pur
%  \item Enseignement + recherche
%  \item Recherche pure
%\end{itemize}
%
%Pour les trois dernières possibilités, se reporter aux définitions ci-dessous.
%Quoi qu'il en soit, il est préférable d'agir au plus tôt durant la thèse en fonction des perspectives de carrière.
%Par exemple, un doctorant qui souhaite devenir chargé de recherche aura intérêt à publier en quantité durant son doctorat.


\end{document}
