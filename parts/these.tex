\chapter{What will it happen to you in thesis?}

\section{Sign PhD}
It's a bit complicated but you'll get there.
You will have to moult back and forth between your desktop printer, the secretariat located in Building A, your supervisors, laboratory management and the graduate school to which you are attached.
Not to mention that little circle your internal laboratory between managers and secretary to the Director of the Laboratory for he gives you a desk, PC, email address ...
One very important thing that you must read is the thesis charter is a contract between you, your manager and the director of the doctoral school.
It says basically that you must do your best in most conditions you must provide your manager, and all overseen by the director of the doctoral school, ensures the smooth running of your thesis.

\section{The thesis grants}
Let us begin by recalling the basics: What is a scholarship?
Well to perform a thesis, it is first necessary funding (this word is also sometimes used as a synonym for "Exchange").
This funding is used to pay for such testing campaigns, hardware or possibly as PhD student ...
Normally, if you read this book for PhD at the moment is that you must already have a scholarship (in any case, hopefully for you) ...
You must therefore tell you that the previous paragraph you do not learn much, but your purse is a purse among many others!
Indeed, there are many.
This section of the GSD has also not pretend to want list exhaustively all types of existing scholarship, but at least we will present the most common funding (for your general culture).
To put it simply, there are three kinds of funding:
\begin{enumerate}
  \item \textbf{public funding:} They come from state agencies or local governments. Include scholarships:
  \begin{itemize}
    \item the Ministry of Research (ANR grants: National Agency for Research, Scholarship MENRT, scholarships for ENS AN ...)
    \item regions (or more generally public authorities)
    \item CNRS (Doctorate Scholarships for Engineers: BDI, ...)
    \item other research organizations (INSERM, INRIA, INRA, INED, CNES, IFREMER, ONERA, ADEME, ANVAR ...)
  \end{itemize}
  \item \textbf{Private financing:} In this case, this is a company that pays the PhD student to work on a particular subject in collaboration with a research laboratory. The most common are scholarships CIFRE (Convention Industrielle de Formation by Research). These are scholarships that have a strong link with the company, and that can lead to a permanent contract.
  \item \textbf{scholarships for foreign students:} Depending on the country, it is possible for some students under certain conditions to obtain a PhD scholarship from their government and / or the French government. If you are in this case, information yourself from your graduate school, some require laboratories to ensure you a minimum wage.
\end{enumerate}

\section{The vocabulary of the thesis}
For you the master of words conf ', postdoc, etc.. mingle happily in your head?
This section will help you then.
\paragraph {Postdoc}
As its name suggests, a postdoc is performed after a PhD.
It allows for a period of 1 to 3 years of rich professional experience.
A postdoc is often required to access the items described below.
\paragraph {Lecturer (TM)}
It is often the status of your supervisor.
It has both a higher teaching load of about 200 h coupled with research.
To become a master conf must be qualified by a committee of the CNU (National Council of Universities) in the given course and get a job ...
\paragraph{Professor University (PU)}
Your supervisor surely belongs to this species.
Unlike the master conf ', has the right to supervise PhD students.
To become a PU, it is necessary to pass the Habilitation Research (HDR) and of course to get a job ...
\oaragraph{Research Fellow and Director of Research}
These are respectively equivalent MC and PU but the teaching load.
They are attached to research organizations such as CNRS, INRIA, INCERM, etc..
\paragraph {Associate Professor (PRAG)} This is a status corresponding to a load of about 400 hours of teaching (not research).
Associate? This term means that the individual is the holder of aggregation, that is to say it has been received in the national competition of the same name.
Aggregation allows for example to teach in preparatory class.
The competition (which includes a theoretical part and a learning part) can prepare one or an SLA that offers a year of preparation framed.

\section{Working after thesis}
What will you do after your PhD?
The first answer is going to be: change your identity papers for the acronym "Dr." as you expected!
But apart from that, what positions are open to you?
\begin{itemize}
  \item engineer / researcher in the industry. The industry is recruiting more doctors, especially in large companies. This hiring is favored by tax charges for the use of a doctor permanent contracts via the Research Tax Credit (CIR). For example, taking advantage of a CIR, the doctor would not cost a sub during the first 2 years.
  \item Teaching pure
  \item + Education Research
  \item Basic Research
\end{itemize}

For the last three options, refer to the definitions below.
Anyway, it is better to act earlier in the thesis based on career prospects.
For example, a doctoral student who wishes to become a researcher's interest to publish much during his PhD.
%\chapter{Que va-t-il t'arriver en thèse ?}\trad
%
%\section{S'inscrire en doctorat}\trad
%C'est un peu compliqué mais tu vas y arriver.
%Tu devras faire moult aller-retours entre ton bureau, l'imprimante, le secrétariat situé au bâtiment A, tes encadrants, la direction du laboratoire et l'école doctorale à laquelle tu es rattaché.
%Sans parler de ce petit cercle vicieux interne à ton laboratoire entre les responsables informatiques et secrétaire du directeur de Laboratoire pour qu'il te donne un bureau, un PC, une adresse mail...
%Une chose très importante que tu devras lire est la charte des thèses qui est un contrat passé entre toi, ton directeur et le directeur de l'école doctorale.
%Il précise en gros que tu devras faire ton maximum dans les meilleures conditions que doit t'offrir ton directeur, et le tout supervisé par le directeur de l'école doctorale, garant du bon déroulement de ta thèse.
%
%\section{Les bourses de thèse}\trad
%Commençons tout d'abord par rappeler les bases : qu'est-ce qu'une bourse ?
%Et bien pour pouvoir effectuer une thèse, il faut d'abord un financement (ce mot est d'ailleurs parfois utilisé comme synonyme de « bourse »).
%Ce financement sert par exemple à payer les campagnes d'essai, le matériel informatique ou éventuellement le thésard aussi...
%Normalement, si tu lis ce recueil pour doctorant en ce moment, c'est que tu dois déjà avoir une bourse (en tout cas, on l'espère pour toi)...
%Tu dois donc te dire que le précédent paragraphe ne t'a pas appris grand-chose, mais ta bourse, c'est une bourse parmi tant d'autres !
%En effet, il en existe une multitude.
%Ce paragraphe du GSD n'a d'ailleurs pas la prétention de vouloir répertorier exhaustivement tous les types de bourses existants, mais nous allons au moins te présenter les financements les plus courants (pour ta culture générale).
%Pour faire simple, il y a 3 sortes de financements :
%\begin{enumerate}
%  \item \textbf{Les financements publics :} Ils proviennent d'organismes de l'état ou des collectivités locales. On peut citer les bourses :
%  \begin{itemize}
%    \item du ministère de la recherche (bourses ANR : Agence Nationale pour la Recherche, bourses MENRT, bourses AN pour normaliens, ...)
%    \item des régions (ou plus généralement des collectivités publiques)
%    \item du CNRS (Bourses de Doctorat pour Ingénieurs : BDI, ...)
%    \item d'autres organismes de recherche (INSERM, INRIA, INRA, INED, CNES, IFREMER, ONERA, ADEME, ANVAR, ...)
%  \end{itemize}
%  \item \textbf{Les financements privés :} Dans ce cas, c'est une entreprise qui paye le thésard pour travailler sur un sujet particulier en collaboration avec un laboratoire de recherche. Les plus répandues sont les bourses CIFRE (Convention Industrielle de Formation par la Recherche). Ce sont des bourses qui présentent un lien fort avec l'entreprise, et qui peuvent conduire à un CDI.
%  \item \textbf{Les bourses pour étudiants étrangers :} Selon les pays, il est possible à certains étudiants étrangers et sous certaines conditions d'obtenir une bourse de thèse auprès de leur gouvernement et/ou du gouvernement français. Si tu es dans ce cas, renseigne-toi auprès de ton école doctorale, certaines demandent aux laboratoires de t'assurer un salaire minimum.
%\end{enumerate}
%
%\section{Le vocabulaire de la thèse}\trad
%Pour toi les mots maître de conf', postdoc, etc. se mélangent allègrement dans ta tête ?
%Cette section va alors t'aider.
%\paragraph{Postdoctorant}
%Comme son nom l'indique, un postdoc est effectué après un doctorat.
%Il permet pendant une période de 1 à 3 ans d'enrichir son expérience professionnelle.
%Un postdoc est souvent requis pour accéder à aux postes décrits ci-après.
%\paragraph{Maître de conférences (MC)}
%C'est bien souvent le statut de ton encadrant.
%Il a à la fois une charge d'enseignement supérieur d'environ 200 h couplée à de la recherche.
%Pour devenir maître de conf, il faut être qualifié par un comité du CNU (Conseil National des Universités) dans la section considérée et bien sûr décrocher un poste...
%\paragraph{Professeur des universités (PU)}
%Ton directeur de thèse appartient sûrement à cette espèce.
%À la différence du maître de conf', lui a le droit de superviser des doctorants.
%Pour devenir PU, il est nécessaire de passer l'Habilitation à Diriger les Recherche (HDR) et aussi bien sûr de décrocher un poste...
%\paragraph{Chargé de recherche et directeur de recherche}
%Ce sont respectivement les équivalents de MC et PU mais sans la charge d'enseignement.
%Ils sont rattachés à des organismes de recherche tels que le CNRS, l'INRIA, INCERM, etc.
%\paragraph{Professeur agrégé (PRAG)} C'est un statut correspondant à une charge d'environ 400h d'enseignement (et pas de recherche).
%Agrégé ? Ce terme signifie que l'individu est titulaire de l'agrégation, c'est-à-dire qu'il a été reçu au concours national du même nom.
%L'agrégation permet par exemple d'enseigner en classe préparatoire.
%Le concours (qui comprend notamment une partie théorique et une partie pédagogique) peut se préparer seul ou à une ENS qui propose une année de préparation encadrée.
%
%\section{Travailler après la thèse}\trad
%Que vas-tu faire après ta thèse ?
%La première réponse qui va de soit est : changer tes papiers d'identités pour obtenir l'acronyme « Dr » que tu attendais tant !
%Mais à part ça, quels postes s'ouvrent à toi ?
%\begin{itemize}
%  \item Ingénieur/chercheur dans l'industrie. L'industrie recrute de plus en plus des docteurs surtout dans les grandes entreprises. Cette embauche est favorisée par des allègements de charges pour l'emploi d'un docteur en Contrat à Durée Indéterminée via le Crédit Impôt Recherche (CIR). A titre d'exemple, en profitant d'un CIR, le docteur ne couterait pas un sous durant les 2 premières années.
%  \item Enseignement pur
%  \item Enseignement + recherche
%  \item Recherche pure
%\end{itemize}
%
%Pour les trois dernières possibilités, se reporter aux définitions ci-dessous.
%Quoi qu'il en soit, il est préférable d'agir au plus tôt durant la thèse en fonction des perspectives de carrière.
%Par exemple, un doctorant qui souhaite devenir chargé de recherche aura intérêt à publier en quantité durant son doctorat.
