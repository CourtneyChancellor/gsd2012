\chapter{L'École Centrale de Nantes et le campus universitaire}

\section{L'enseignement supérieur français}\trad
Comme pour les autres pays européens, l'enseignement supérieur est basé sur le système « LMD » : Licence Master Doctorat.
En France, il n'y a pas que les universités, mais il existe également des Grandes Écoles, dites écoles d'ingénieurs, comme l'École Centrale Nantes.
Pour débuter un doctorat en France, on peut avoir suivi soit un cursus universitaire (en passant donc par un master recherche) soit venir d'une école d'ingénieurs.


\section{Tout sur l'École Centrale de Nantes}\trad
\subsection{L'école en quelques chiffres}\trad
L'école a été fondée il y a plus de 90 ans.
D'abord appelée l'IPO (Institut Polytechnique de l'Ouest depuis 1919), puis l'ENSM (École Nationale de Mécanique à partir de 1947), elle est rattachée au groupe des Écoles Centrales en 1991.
Elle était initialement basée en centre ville, elle est arrivée sur le campus du Tertre en 1977.
\todo{Vérifier ces chiffres}
\paragraph{Les gens}
\begin{itemize}
  \item 1850 étudiants (1200 élèves-ingénieurs, 250 élèves-ingénieurs en formation continue et par apprentissage, 200 étudiants de Master, 200 doctorants),
  \item 550 chercheurs, enseignants-chercheurs et personnels de recherche,
  \item 150 personnels administratifs et techniques.
\end{itemize}
\paragraph{Le Lieu}
\begin{itemize}
  \item 40000 m\textsuperscript{2} de locaux,
  \item 1 campus de 16 ha,
  \item 5 laboratoires labellisés CNRS.
\end{itemize}
\paragraph{Centrale Nantes en France et dans le monde}
\begin{itemize}
  \item 11\textsuperscript{e} meilleure École d'Ingénieurs de France (classement 2010),
  \item 115 universités partenaires dans 40 pays.
\end{itemize}

\subsection{Les grandes têtes de l'école}\trad
Voici une petite liste (non exhaustive) de personnalités importantes du campus, que vous serez amenés à croiser durant vos 3 ans de thèse.
\todo{Compléter les noms et vérifier les nouveaux intitulés des fonctions}

\newcommand{\largcol}{0.29\textwidth}

\newcommand{\pers}[4]{%
\begin{tabular}{m{\largcol}}%
\ifthenelse{\equal{#1}{}}{\todo{Photo}}{}%
%\ifthenelse{\equal{#1}{}}{\todo{Photo}}{\includegraphics[width=2.5cm]{#1}}%
%\includegraphics{#1}
\\ \textbf{#2} \\ \textit{(#3)} \\ #4
\end{tabular}}

\noindent
\begin{tabular}{*{3}{p{\largcol}}}
  \pers{}{}{Directeur de l'école}{C'est notre chef à tous. On lui doit allégeance, même si en pratique, on a rarement affaire à lui.}&
  \pers{}{}{Directeur des études}{Comme son statut l'indique, il est davantage en lien avec les élèves. Amis moniteurs, si vous avez des problèmes avec vos enseignements, c'est peut-être à sa porte qu'il faut aller frapper.}&
  \pers{}{}{Directeur de la recherche}{C'est notre boss à nous. Il gère la politique de recherche de l'école (les projets, l'argent, les diverses activités, \dots) et peut s'occuper de nous en cas de [gros] problèmes avec notre thèse.}\\
  \pers{}{}{Secrétaire général}{Il s'occupe des grandes décisions administratives qui nous concernent (oui, la paperasse de l'inscription, c'est en partie pour son service\dots). Véritable encyclopédie des textes officiels, il pourra répondre aux questions auxquelles tout autre secrétaire aura jeté l'éponge.}&
  \pers{}{}{Directeur de la communication}{Lui et son équipe gèrent la diffusion des informations et l'événementiel sur le campus. Son travail dépend notamment de votre participation, alors n'hésitez pas à l'informer de tout ce qu'il se passe dans votre labo.}&
  \pers{}{}{Directeur adjoint}{C'est l'homme de terrain. Il met à exécution les grandes décisions « concrètes » de l'administration. Vous recevrez notamment pas mal de mails durant l'année sur d'éventuels travaux, changement d'organisation, etc. de sa part.}
\end{tabular}

\subsection{Les étudiants}\trad
L'école renferme une population d'étudiants assez hétéroclite.
Il y a :
\paragraph{Les élèves-ingénieurs (EI1, EI2 et EI3)} Ils sont admis à l'école sur concours après une classe prépa (bac+2) et constituent la majorité des étudiants du campus et sont présents pour 3 ans. Très actifs sur le plan associatif, ils animeront le campus tout au long de l'année. Les premières années vivent dans la résidence près du campus. Depuis peu, certains élèves ingénieurs travaillent en alternance au même titre que les ITII (voir plus bas).
\paragraph{Les Masters (M1 et M2)} Originaires d'un parcours universitaire tiers, les masters partagent souvent leurs cours avec les élèves ingénieurs. On peut notamment citer les Master Erasmus-Mundus attirant des étudiants du monde entier!
\paragraph{Les ingénieurs par alternance ITII Pays de la Loire (1ère, 2ème et 3ème année)} Il s'agit d'étudiants en formation par alternance (2 semaines en cours, 2 semaines en entreprise). L'ITII Pays de la Loire est un organisme de formation extérieur à l'École Centrale Nantes, cependant les cours se font sur le campus. Officiellement, ce ne sont pas des étudiants. Dans la pratique, ils sont exactement comme les autres, à la différence qu'ils ne sont présents que 2 semaines sur 4.
\paragraph{Les doctorants (D1, D2 et D3)} C'est toi.

\subsection{Les labos de l'école}\trad
Le campus de l'École Centrale comprend 5 labos (voir le plan du campus) :
\paragraph{Le GeM} De son vrai nom « Institut de Recherche en Génie civil et Mécanique ». Le Gem est un labo installé sur 3 sites : l'École Centrale, la fac de science et l'IUT de St Nazaire. On y fait de la simulation mécanique, du génie civil, des études de matériaux, de la dynamique rapide (crash-test)\dots Parmi tous les thèmes de recherche (et il y en a beaucoup), deux ressortent : La méthode X-FEM (méthode des éléments finis enrichis) et la PGD (méthode de décomposition de variables pour résoudre des problèmes beaucoup plus rapidement). D'autre part, un domaine d'application particulièrement en vogue ces temps ci est celui des matériaux composites.
\paragraph{Le LHEEA} (« Laboratoire de recherche en Hydrodynamique, Énergétique et Environnement Atmosphérique ») C'est sans doute le laboratoire qui possède le matériel expérimental le plus impressionnant du campus : un bassin de carène (pour tracter des maquettes de bateaux), un bassin de houle (pour faire des vagues), une soufflerie\dots Le LHEEA est morcelé en plusieurs équipes dont les principaux thèmes sont l'hydrodynamique et génie océanique (projet « SEAREV » de récupération d'énergie des vagues), l'énergétique des moteurs à combustion interne, l'étude de la dynamique de l'atmosphère en milieu urbain.
\paragraph{L'IRCCyN} (« Institut de Recherche en Communications et Cybernétique de Nantes ») Principalement installé dans le bâtiment S (le bâtiment relativement moderne, de forme ovale), l'IRCCyN concentre douze thématiques de recherche très variées. On y travaille par exemple sur la commande, la robotique, la conception, la fabrication additive, l'optimisation multidisciplinaire, les systèmes temps réels et même\dots la psychologie ! L'IRCCyN est réparti sur 4 sites de Nantes : l'École Centrale, l'École des Mines, l'École Polytechnique et l'IUT.
\paragraph{Le CERMA} (Centre de Recherche Méthodologique d'Architecture). Chapeauté depuis peu par l'École Centrale, le CERMA est avant tout un laboratoire dépendant de l'École Nationale Supérieure d'Architecture de Nantes. On y travaille notamment sur toute la physique liée l'architecture et l'environnement urbain.
\paragraph{Le laboratoire de mathématique Jean Leray} On n'en entend pas trop parler, mais il est là, caché dans une partie du bâtiment E ! En fait, c'est parce que l'essentiel se trouve à la fac de sciences et seul un petit bout est à l'École Centrale Nantes. La recherche à Centrale se concentre sur l'étude des équations aux dérivées partielles non-linéaire avec des domaines d'application pouvant aller des problèmes dynamique de populations à l'étude des changement d'état de matériaux supraconducteurs.

\subsection{Plan du site de l’École Centrale Nantes}\trad
\todo{Inclure le plan}

\section{Sport and culture on campus}
\subsection {Places to practice sports}
Do you wanna romp? Go out of your 10 m2 office?  Let's practice sport! Three structures are at your disposal around the campus.

\subsubsection{The Athletic Association Centrale Nantes (AS)}
It is the association managed by the sport association of th engineering students (BDS) from Centrale. Even if it is mainly dedicated to engineering students, it is commonly open to master and phd students. A membership fee can be asked to participate at some activities.
\paragraph{Contacts}
\begin{itemize}
  \item Email address of the sport association: \texttt{bds@ec-nantes.fr}
  \item Office: 1\textsuperscript{st} floor of L Building
  \item \texttt{http://bds.campus.ec-nantes.fr}
\end {itemize}
\paragraph{How to apply}
\begin{itemize}
  \item A cheque \EUR{55}, for the membership fee
  \item A medical certificate \footnote{-filled meadows certificates are available at the office of BDS} non-cons-indication (if you specify the competition)
\end{itemize}

\paragraph{Advantage}
\begin{itemize}
  \item [$+$] Many sports (17)
  \item [$+$] Possibility of academic competitions,
  \item [$+$] Diversity of levels,
  \item [$+$] Some offer sports training with teacher
  \item [$+$] You may find yourself playing with your own students!
\end{itemize}
\paragraph{Drawback}
\begin{itemize}
  \item [$-$] You may find yourself playing with your own students,
  \item [$-$] Sometimes too crowded,
  \item [$-$] Requires consistency in training.
\end{itemize}

\paragraph{Warning} The competitions are usually held on Thursday afternoon (often at the same time as seminars).


\paragraph{Some events:}
\begin{itemize}
  \item Outside Centrale: The Inter-Centrales, Challenge Lyon, BER (Brest), CAS (Supélec Rennes), the TOSS (Supélec Paris).
  \item In Centrale:  4 balls Tournament (T4B) 3 rackets Tournament (T3R), Inter-groups.
\end{itemize}

\subsubsection {Association Sportive du Personnel (ASP)}
It is the staff association of the Ecole Centrale (teachers, administrators, technicians, ... and PhD!). The association currently offers only four sports (futsal, tennis, gym and badminton) but is open to any contribution to develop other sports.

\paragraph{} during lunch time : 12:00 - 13:30.
\paragraph{Contacts}
\begin {itemize}
  \item Chairperson: \texttt{jean-paul.bouganne@ec-nantes.fr} (02 40 37 25 92)
  \item Intranet site of ECN
\end {itemize}
\paragraph{How to apply}
\begin{itemize}
  \item A cheque \EUR {13} (+ \EUR{13} for tennis \footnote {The tennis requires to pay a license FFT}).
  \item A medical certificate of non-cons-indication.
\end{itemize}

\paragraph{Advantage}
\begin{itemize}
  \item Cheap,
  \item Often fewer people,
  \item During lunch time.
\end{itemize}
\paragraph{Drawback}
\begin{itemize}
  \item During lunch time,
  \item Few sports (4),
  \item No competition possible (except for tennis!)
\end{itemize}

\subsubsection {The SUAPS}
SUAPS is the university sport association. It is located in the gymnase in front of the university restaurant "Le Tertre".
Even if you are not a student of the university, you can still sign up at SUAPS.
This is usually a good alternative if the sport you want to practice is not available in Centrale.

\paragraph{Warning} If, as a PhD student, you are not part of the university, you will be considered as "external student."
Registration will cost more (\EUR{50} instead of \EUR{35}), and you may have not access to some popular sports.
Visit the website for more information on SUAPS.

\paragraph{Advantage}
\begin{itemize}
  \item Many sports (47), including rare ones
  \item Lot of equipments,
  \item From beginners to experienced.
\end{itemize}

\paragraph{Weaknesses}
\begin{itemize}
  \item Sometimes too crowded,
  \item Off Centrale campus,
  \item You can be considered as "external student" in the university, and therefore not a priority in some sports.
\end{itemize}

\subsection {Cultural Associations Student}
\subsubsection {At the Ecole Centrale}
\paragraph{AED} Association of PhD Students from the Ecole Centrale Nantes. Its purpose is to connect PhD students from the differents laboratories located on the Centrale campus. Thus it organizes events, professional, cultural or even festive ones.
You can join this association on the Facebook group "ECN-AED" to keep up with the news. Website: \url{http://website.ec-nantes.fr/aed/}
\paragraph{AMAP Ecole Centrale} BDS and Central Green ECN offer local and organic fruits and vegetables, delivered each week. For more information, contact BDS.



%\section{Sport et culture sur le campus}\trad
%\todo{Vérifier les infos de cette section}
%\subsection{Où faire du sport ?}\trad
%Envie de te défouler ? De sortir de tes 10 m\textsuperscript{2} de bureau ?
%Va faire du sport ! Trois structures te permettent cela.
%
%\subsubsection{L'Association sportive de Centrale Nantes (AS)}
%C'est l'association gérée par le Bureau Des Sports (BDS) des étudiants de Centrale. En tant qu'étudiants, les doctorants peuvent s'inscrire à l'AS.
%\paragraph{Contacts}
%\begin{itemize}
%  \item L'adresse courriel du bureau des sports : \texttt{bds@ec-nantes.fr}
%  \item Bureau au 1\textsuperscript{er} étage du bâtiment L
%  \item \texttt{http://bds.campus.ec-nantes.fr}
%\end{itemize}
%\paragraph{Modalités d'inscription}
%\begin{itemize}
%  \item Un chèque de \EUR{55},
%  \item Un certificat médical\footnote{Des certificats prés-remplis sont disponibles au bureau du BDS} de non-contre-indication (précisez si vous faites de la compétition)
%\end{itemize}
%
%\paragraph{Points forts}
%\begin{itemize}
%  \item[$+$] Beaucoup de sports (17),
%  \item[$+$] Possibilité de faire des compétitions universitaires,
%  \item[$+$] Diversité de niveaux,
%  \item[$+$] Certains sports proposent des entrainements avec professeur,
%  \item[$+$] Tu peux te retrouver à jouer avec tes propres élèves !
%\end{itemize}
%\paragraph{Points faibles}
%\begin{itemize}
%  \item[$-$] Tu peux te retrouver à jouer avec tes propres élèves,
%  \item[$-$] Parfois trop de monde,
%  \item[$-$] Nécessite une régularité dans les entraînements.
%\end{itemize}
%
%\paragraph{Attention} Les compétitions ont généralement lieu les jeudis après-midi (souvent en même temps que des séminaires).
%Tout comme les entraînements, elles demandent un engagement particulier vis-à-vis vos équipes et de vos adversaires.
%Avant de t'engager, vérifie bien que tes obligations de doctorants te permettent de participer régulièrement à ces compétitions.
%
%\paragraph{Quelques événements tout au long de l'année}
%\begin{itemize}
%  \item Hors de Centrale : Les Inter-Centrales, le Challenge Lyon, le TEB (Brest), les CAS (Supélec Rennes), le TOSS (Supélec Paris).
%  \item Dans Centrale : Tournoi des 4 ballons (T4B), tournoi des 3 raquettes (T3R), Inter-groupes.
%\end{itemize}
%
%\subsubsection{L'Association Sportive du Personnel (ASP)}
%C'est l'association du personnel de l'École Centrale (professeurs, administratifs, techniciens, etc. et doctorants !). L'association ne propose que quatre sports (foot en salle, Tennis, musculation et badminton) mais ne demande qu'à évoluer. Si tu es motivé pour monter un sport, n'hésite surtout pas ! Pour info, il y avait avant du ping-pong et de la danse.
%
%\paragraph{Créneaux horaires} 12h--13h30.
%\paragraph{Contacts}
%\begin{itemize}
%  \item Président : \texttt{jean-paul.bouganne@ec-nantes.fr} (02 40 37 25 92)
%  \item Intranet du site de l'ECN
%\end{itemize}
%\paragraph{Modalités d'inscription}
%\begin{itemize}
%  \item Un chèque de \EUR{13} (+ \EUR{13} pour le tennis\footnote{Le tennis nécessite de payer une licence FFT}).
%  \item Un certificat médical de non-contre-indication.
%\end{itemize}
%
%\paragraph{Points forts}
%\begin{itemize}
%  \item Pas cher,
%  \item Souvent moins de monde,
%  \item Idéal pour aller se défouler sans prise de tête
%\end{itemize}
%\paragraph{Points faibles}
%\begin{itemize}
%  \item Peu de sports (4),
%  \item Pas de compétition possible (sauf pour le tennis !)
%\end{itemize}
%
%\subsubsection{Le SUAPS}
%Le SUAPS est l'organisme responsable des sports à la fac de Nantes.
%C'est une GROSSE structure basée dans le gymnase en face du resto U.
%Même si tu n'es pas étudiant de la fac, tu peux quand même t'inscrire au SUAPS.
%C'est en général une bonne alternative si le sport que tu désires n'est pas disponible sur le campus de Centrale.
%
%\paragraph{Attention} Si, en tant que doctorant, tu n'es pas inscrit à l'université, tu seras considéré comme « étudiant extérieur ».
%L'inscription te coûtera plus cher (\EUR{50} au lieu de \EUR{35}), et certains sports « populaires » ne te seront pas autorisés.
%Rendez-vous sur le site du SUAPS pour plus d'informations.
%
%\paragraph{Points forts}
%\begin{itemize}
%  \item Beaucoup de sports (47), dont des sports « rares »,
%  \item Beaucoup de matériel,
%  \item Grande diversité de niveaux,
%  \item Propose des créneaux pour la piscine.
%\end{itemize}
%\paragraph{Points faibles}
%\begin{itemize}
%  \item Parfois trop de monde,
%  \item Hors du campus,
%  \item Tu peux être considéré comme « étudiant extérieur » à la fac, et donc non-prioritaire sur certains sports.
%\end{itemize}
%
%\subsection{Associations culturelles étudiantes}\trad
%\subsubsection{À l'École Centrale}
%\paragraph{AED} Association des Élèves en Doctorat de l'École Centrale Nantes. Elle fédère tous les doctorants présents sur le site de l'ECN et organise des événements tant festifs que scientifiques. Rejoins-nous sur le groupe Facebook « AED-ECN » pour voir le fil des infos. Site internet : \url{http://website.ec-nantes.fr/aed/}
%\paragraph{AMAP Ecole Centrale} Le BDS et Centrale Vert de l'ECN propose à la rentrée 2011 des paniers de légumes bio, distribués toutes les semaines. Tu auras de plus amples informations à la rentrée 2011-2012, ou par mail.
%
%D'autre part, les bureaux des élèves (BDE, BDA, BDS) organisent en Septembre une soirée de rentrée des associations. Tu pourras y découvrir tous les clubs et éventuellement t'y inscrire ! Renseigne-toi sur la date.
%
%\subsubsection{Sur le campus universitaire}
%\paragraph{Association « Des pieds dans la tête »} Propose des ciné-philo ; le but de l'association est de sortir la philosophie du cadre de l'enseignement et d'ouvrir des débats plus larges par exemple sur la liberté. Site : \url{http://depidalate.free.fr/}
%\paragraph{Maison des Jeux} Tous les mercredis midi au TU, tu pourras découvrir et participer à des jeux de société et jeux en bois.



\section{Trouver un logement}\trad
\paragraph{Cité universitaires} En tant que doctorant tu es normalement prioritaire pour obtenir un logement du CROUS. Le mieux est de se rendre directement  à l'accueil du CROUS (2 bd Guy Mollet arrêt de TRAM Faculté) et se référer au site internet du CROUS www.crous-nantes.fr. ATTENTION, il y a beaucoup de demandes, il est donc préférable de s'y prendre plusieurs mois à l'avance (au mois d'Avril pour Septembre) !
\paragraph{Colocation} Si vivre à plusieurs te tente, tu peux trouver sur www.appartager.fr plein d'annonces gratuites pour partager les frais et faire des rencontres très rapidement.
\paragraph{Location d'appartement} Pour trouver un appartement à louer, il y a deux possibilités : les agences immobilières (ATTENTION, il faut généralement payer des frais d'agence pouvant s'élever à 1 mois de loyer), les annonces de particuliers (leboncoin.fr, de particulier à particulier pap.fr et/ou le CROUS). Penses également à acheter le supplément immobilier du journal Ouest France publié tous les samedi !
\paragraph{Achat d'appartement} Bon ben là, il faut y réfléchir mais si tu as de l'argent de coté ou que tu aimes harceler ton banquier, ça peut être un bon investissement ; tu vas surement rester 3 ans sur place !!

Pour le lieu de ton logement tu feras face au dilemme classique :
\begin{itemize}
  \item Habiter au centre ville, près de toutes les animations étudiantes, facilement accessible par le tram pour te rendre au boulot. Mais tu risques d'avoir un petit logement cher avec des difficultés pour garer ta voiture.
  \item Habiter une petite maison pas très chère en banlieue au milieu de la nature avec jardin et barbecue mais tu devras investir dans une voiture et faire une croix sur le troisième verre pour rentrer chez toi...
  \item Dormir dans ton bureau (tu remarqueras comme la moquette est confortable... et propre !!)
\end{itemize}

Lors de ta première année de thèse, tu peux avoir droit aux APL (Aides Personnalisées au Logement) puisqu'elles se basent sur les revenus que tu as obtenus l'année précédente pour calculer ton aide (penses à valoriser ton statut d'étudiant). Se renseigner sur le site \url{www.caf.fr} de la Caisse d'allocation familiale qui délivre cette aide personnalisée. En tout cas, mieux vaut faire la demande, ça ne mange pas de pain.

\section{Manger près de Centrale}\trad
\subsection{Les restaurants universitaires (ou « RU »)}
Les « restos U » auxquels tu peux avoir facilement accès à pied depuis Centrale se comptent au nombre de trois : le Tertre, le Rubis et la Cafétéria du pôle étudiant. Il en existe d'autres à Nantes et tu trouveras toutes les informations les concernant sur le site du CROUS de Nantes (\url{http://www.crous-nantes.fr}). Le tarif d'un plateau-repas au RU se situe aux alentours des 3 euros (tarif 2011 – 2012).
Le payement s'effectue exclusivement grâce à ta carte d'étudiant Pass'Sup, qui fait aussi office de carte Moneo.

Tu peux créditer ta carte Pass' Sup dans n'importe quels bureaux de poste, banques, cabines téléphoniques, halls d'accueil de restaux U, etc... arborant le logo « Moneo », ou encore en ligne sur \url{www.moneo.net} !!

\subsubsection{Le Tertre}
Le Tertre se situe juste en face de l'arrêt de tramway Facultés. Il comporte deux étages et permet un choix relativement varié de cuisines.
On y trouve au rez-de-chaussée une sandwicherie ainsi qu'un coin café et à l'étage plusieurs pôles thématiques (cuisine du monde, poisson, grill, pizza, etc...).

Ouverture : Le midi (11h00-13h30) du lundi au vendredi.

\subsubsection{Le Rubis}
L'intérêt principal de ce restaurant est qu'il est en général moins pris d'assaut que son homologue le Tertre et qu'en plus il est ouvert le soir (avec une sympathique petite musique de fond bien agréable pour digérer toutes les formules et équations de la journée).
Ainsi, tu pourras travailler ta thèse jusqu'à 19h00, manger au RU et enfin revenir pour poursuivre ton travail (jusqu'à la fermeture de l'école à 22h00 pile)

Ouverture : Le midi (11h00-13h30) et le soir (18h30-20h00) du lundi au samedi midi.

\subsection{La Cafétéria du pôle étudiant}
La cafétéria du pôle étudiant se situe entre les Facs de Droit et de Lettres.
Elle est ouverte en continu de 8h30 à 17h30 et l'ambiance est assez "chaleureuse" (ambiance Lettre et Droit quoi...).
On y sert des repas "chauds" en continu (paninis, tapas, etc...) et tu y trouveras aussi crudités et viennoiseries.

Ouverture : Du lundi au Vendredi de 9h30 à 21h00.

\subsection{Le bâtiment L (la K-fet' de l'école)}
Il fut un temps pas si lointain où le bâtiment L de l'école abritait une cafétéria permettant aux élèves et au personnel de s'y restaurer sans quitter le campus.
Cette K-fet' ayant fermé ses portes pour des raisons administratives en 2012, il n'est plus possible de s'y rendre le midi.
Cependant, la rumeur court qu'elle pourrait renaître de ses cendres ; affaire à suivre...

\subsection{La baraque à frites}
Quand les RU sont fermés ou quand le temps te manque, tu trouveras en général une camionnette-snack garée au niveau du rond point de l'école.
Double-américain avec surplus de ketchup et frites au menu !!

Ouverture : Le midi du lundi au vendredi.

\subsection{Le Fac' Food}

Le Fac' Food est, tu l'as deviné, une sandwicherie. Elle est située en face de l'école et tu y trouveras sandwichs, burgers en tous genres, paninis, américains, etc...

Ouverture : Du lundi au vendredi de 8h00 à 15h00.

\subsection{Pizzeria-brasserie « Le Café Gourmand »}

Cette brasserie jouit d'une bonne réputation sur le net et elle est pourvue d'une grande terrasse-jardin à l'arrière.
Formule "Pizza - Café Gourmand" à 14 euros.

Ouverture : Le midi du lundi au vendredi et le jeudi soir (pizza à emporter)

\subsection{Bar-brasserie « Les Facultés »}
La brasserie Les Facultés est située juste en face de l'école à côté du Fac' Food (même enseigne).
Tu pourras y manger en terrasse et tu y trouveras également un bureau de tabac.

Ouverture :     Du lundi au vendredi de 7h00 à 20h00 et le samedi de 7h30 à 14h00.

\subsection{Bar-brasserie « Le Campus »}
Cette petite brasserie avec terrasse située tout près de Centrale, rue du Fresche Blanc, t'accueillera avec des formules midi de 7 à 15 euros (+ buffet à volonté).

Ouverture :     Du lundi au vendredi de 7h00 à 17h00.


\section{Aide aux étudiants étrangers}\trad
Tu es un chercheur étranger à Nantes ?
L'association « Chercheurs Étrangers à Nantes » constitue un guichet unique en matière d'information et d'accueil.
Parmi les services proposés, on retrouve :
\begin{itemize}
  \item L'aide à la préparation du séjour,
  \item Des services personnalisés pour ton arrivée (information sur la vie quotidienne, transport, plans, scolarisation des enfants...),
  \item Un appui aux démarches administratives de séjour (constitution et dépôt du dossier de demande de titre de séjour, informations pratiques sur les taxes, salaires, assurances, visite médicale, banque, etc...).
  \item Des cours de français en tant que langue étrangère (à faible coût), niveaux débutant et avancé
\end{itemize}

Par ailleurs, tu as envies de participer, de créer ou de t'engager ?
Les associations à Nantes sont à ton écoute !
Tu trouveras des associations impliquées dans plusieurs secteurs d'activité.
N'hésite pas à visiter le site internet de l'Université de Nantes (rubrique vie associative) pour plus d'informations.

\subsection{Association de regroupement culturel}
Les associations des étudiants étrangers ont pour but d'être un repère pour un étudiant étranger à Nantes et de promouvoir leurs cultures (danse, alimentation, musique, ...).
On retrouve l'association des étudiants libanais à Nantes \textbf{ANADYL} (\texttt{asso\_anadyl@yahoo.fr}), l'association des étudiants guinéens à Nantes \textbf{AGEN} (\texttt{agen.nantes@yahoo.com}), l'association des étudiants vietnamiens à Nantes \textbf{AEVN} (\texttt{contact@aevn.fr}), l'association des étudiants gabonais à Nantes \textbf{GaboNantes} (\texttt{gabo\_nantes2@yahoo.fr}), l'association des étudiants italiens à Nantes \textbf{Issimo.it} (\texttt{issimo.nantes@gmail.com}) et l'association des étudiants Roumains à Nantes \textbf{Roumanie d'ici} (\texttt{roumaniedici@yahoo.fr}).
N'hésite pas à te faire connaître auprès de ces associations.
Et d'ailleurs si tu es Français n'hésite pas non plus à te faire passer pour un ERASMUS car en France, il n'y a pas d'associations d'étudiants français !
