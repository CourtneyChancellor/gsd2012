\section{Sport and culture on campus}
\subsection {Places to practice sports}
Do you wanna romp? Go out of your 10 m2 office?  Let's do some sports! Three avenues are at your disposal around the campus.

\subsubsection{The Athletic Association Centrale Nantes (AS)}
Managed by the sports association of engineering students (BDS) of Centrale, this organisation is open to all master and PhD students. Some activities require a membership fee in order to participate.\paragraph{Contacts}
\begin{itemize}
  \item Email address of the BDS: \texttt{bds@ec-nantes.fr}
  \item Office: 1\textsuperscript{st} floor of L Building
  \item \texttt{http://bds.campus.ec-nantes.fr}
\end {itemize}
\paragraph{How to apply}
\begin{itemize}
  \item A cheque \EUR{55}, to cover the membership fee
  \item A medical certificate of a physical exam for insurance purposes\footnote{-filled meadows certificates are available at the office of BDS} non-cons-indication (if you specify the competition)
\end{itemize}

\paragraph{Pros}
\begin{itemize}
  \item [$+$] A variety of sports (17)
  \item [$+$] Possibility of intramural competitions
  \item [$+$] A diverse range of levels (intermediate, beginner, etc),
  \item [$+$] Some teams offer training with a coach
  \item [$+$] You may find yourself on the field with your own students!
\end{itemize}
\paragraph{Cons}
\begin{itemize}
  \item [$-$] You may find yourself on the field with your own students,
  \item [$-$] Can get crowded
  \item [$-$] Commitment and consistency required for training.
\end{itemize}

\paragraph{Warning} The competitions are usually held on Thursday afternoon (often at the same time as seminars).


\paragraph{Some events:}
\begin{itemize}
  \item Outside Centrale: The Inter-Centrales, Challenge Lyon, BER (Brest), CAS (Supélec Rennes), the TOSS (Supélec Paris).
  \item In Centrale:  4 balls Tournament (T4B) 3 rackets Tournament (T3R), Inter-groups.
\end{itemize}

\subsubsection {Association Sportive du Personnel (ASP)}
This association is for the staff of Ecole Centrale (teachers, administrators, technicians, ... and PhDs!). The association currently offers only four sports (futsal, tennis, gym and badminton) but is open to any contribution to develop other sports.

\paragraph{} during lunch time : 12:00 - 13:30.
\paragraph{Contacts}
\begin {itemize}
  \item Chairperson: \texttt{jean-paul.bouganne@ec-nantes.fr} (02 40 37 25 92)
  \item Internet site of ECN
\end {itemize}
\paragraph{How to apply}
\begin{itemize}
  \item A cheque \EUR {13} (+ \EUR{13} for tennis \footnote {Tennis requires the payment of a license FFT}).
  \item A medical certificate of non-cons-indication. (Standard physical exam)
\end{itemize}

\paragraph{Pros}
\begin{itemize}
  \item Cheap,
  \item Often fewer people,
  \item During lunch time.
\end{itemize}
\paragraph{Cons}
\begin{itemize}
  \item During lunch time,
  \item Few sports (4),
  \item No competition possible (except for tennis!)
\end{itemize}

\subsubsection {The SUAPS}
SUAPS is the university sport association. It is located in the gymnasium across the street from the university restaurant "Le Tertre".
Even if you are not a student of the university, you can still sign up at SUAPS for a marginal fee.
This is usually the best alternative if the sport you want to play is not available at Centrale.

\paragraph{Warning} If, as a PhD student, you are not part of the University of Nantes, you will be considered as "external student."
Registration will cost more (\EUR{50} instead of \EUR{35}), and you may have not access to some popular sports.
Visit the website for more information on SUAPS.

\paragraph{Pros}
\begin{itemize}
  \item Many sports (47), including rare ones
  \item Lot of equipment,
  \item From beginners to experienced.
\end{itemize}

\paragraph{Cons}
\begin{itemize}
  \item Sometimes too crowded,
  \item Off Centrale campus,
  \item You can be considered as "external student" to the university, and therefore not a priority in some sports.
\end{itemize}

\subsection {Cultural Associations Student}
\subsubsection {At the Ecole Centrale}
\paragraph{AED} Association of PhD Students from the Ecole Centrale de Nantes. Its purpose is to connect PhD students from the differents laboratories located on the Centrale campus, often organizing events... professional, cultural or even festive ones.
You can join this association on the Facebook group "ECN-AED" to keep up with the news. Website: \url{http://website.ec-nantes.fr/aed/}
\paragraph{AMAP Ecole Centrale} BDS and Central Green ECN offer local and organic fruits and vegetables, delivered each week. For more information, contact BDS.



%\section{Sport et culture sur le campus}\trad
%\todo{Vérifier les infos de cette section}
%\subsection{Où faire du sport ?}\trad
%Envie de te défouler ? De sortir de tes 10 m\textsuperscript{2} de bureau ?
%Va faire du sport ! Trois structures te permettent cela.
%
%\subsubsection{L'Association sportive de Centrale Nantes (AS)}
%C'est l'association gérée par le Bureau Des Sports (BDS) des étudiants de Centrale. En tant qu'étudiants, les doctorants peuvent s'inscrire à l'AS.
%\paragraph{Contacts}
%\begin{itemize}
%  \item L'adresse courriel du bureau des sports : \texttt{bds@ec-nantes.fr}
%  \item Bureau au 1\textsuperscript{er} étage du bâtiment L
%  \item \texttt{http://bds.campus.ec-nantes.fr}
%\end{itemize}
%\paragraph{Modalités d'inscription}
%\begin{itemize}
%  \item Un chèque de \EUR{55},
%  \item Un certificat médical\footnote{Des certificats prés-remplis sont disponibles au bureau du BDS} de non-contre-indication (précisez si vous faites de la compétition)
%\end{itemize}
%
%\paragraph{Points forts}
%\begin{itemize}
%  \item[$+$] Beaucoup de sports (17),
%  \item[$+$] Possibilité de faire des compétitions universitaires,
%  \item[$+$] Diversité de niveaux,
%  \item[$+$] Certains sports proposent des entrainements avec professeur,
%  \item[$+$] Tu peux te retrouver à jouer avec tes propres élèves !
%\end{itemize}
%\paragraph{Points faibles}
%\begin{itemize}
%  \item[$-$] Tu peux te retrouver à jouer avec tes propres élèves,
%  \item[$-$] Parfois trop de monde,
%  \item[$-$] Nécessite une régularité dans les entraînements.
%\end{itemize}
%
%\paragraph{Attention} Les compétitions ont généralement lieu les jeudis après-midi (souvent en même temps que des séminaires).
%Tout comme les entraînements, elles demandent un engagement particulier vis-à-vis vos équipes et de vos adversaires.
%Avant de t'engager, vérifie bien que tes obligations de doctorants te permettent de participer régulièrement à ces compétitions.
%
%\paragraph{Quelques événements tout au long de l'année}
%\begin{itemize}
%  \item Hors de Centrale : Les Inter-Centrales, le Challenge Lyon, le TEB (Brest), les CAS (Supélec Rennes), le TOSS (Supélec Paris).
%  \item Dans Centrale : Tournoi des 4 ballons (T4B), tournoi des 3 raquettes (T3R), Inter-groupes.
%\end{itemize}
%
%\subsubsection{L'Association Sportive du Personnel (ASP)}
%C'est l'association du personnel de l'École Centrale (professeurs, administratifs, techniciens, etc. et doctorants !). L'association ne propose que quatre sports (foot en salle, Tennis, musculation et badminton) mais ne demande qu'à évoluer. Si tu es motivé pour monter un sport, n'hésite surtout pas ! Pour info, il y avait avant du ping-pong et de la danse.
%
%\paragraph{Créneaux horaires} 12h--13h30.
%\paragraph{Contacts}
%\begin{itemize}
%  \item Président : \texttt{jean-paul.bouganne@ec-nantes.fr} (02 40 37 25 92)
%  \item Intranet du site de l'ECN
%\end{itemize}
%\paragraph{Modalités d'inscription}
%\begin{itemize}
%  \item Un chèque de \EUR{13} (+ \EUR{13} pour le tennis\footnote{Le tennis nécessite de payer une licence FFT}).
%  \item Un certificat médical de non-contre-indication.
%\end{itemize}
%
%\paragraph{Points forts}
%\begin{itemize}
%  \item Pas cher,
%  \item Souvent moins de monde,
%  \item Idéal pour aller se défouler sans prise de tête
%\end{itemize}
%\paragraph{Points faibles}
%\begin{itemize}
%  \item Peu de sports (4),
%  \item Pas de compétition possible (sauf pour le tennis !)
%\end{itemize}
%
%\subsubsection{Le SUAPS}
%Le SUAPS est l'organisme responsable des sports à la fac de Nantes.
%C'est une GROSSE structure basée dans le gymnase en face du resto U.
%Même si tu n'es pas étudiant de la fac, tu peux quand même t'inscrire au SUAPS.
%C'est en général une bonne alternative si le sport que tu désires n'est pas disponible sur le campus de Centrale.
%
%\paragraph{Attention} Si, en tant que doctorant, tu n'es pas inscrit à l'université, tu seras considéré comme « étudiant extérieur ».
%L'inscription te coûtera plus cher (\EUR{50} au lieu de \EUR{35}), et certains sports « populaires » ne te seront pas autorisés.
%Rendez-vous sur le site du SUAPS pour plus d'informations.
%
%\paragraph{Points forts}
%\begin{itemize}
%  \item Beaucoup de sports (47), dont des sports « rares »,
%  \item Beaucoup de matériel,
%  \item Grande diversité de niveaux,
%  \item Propose des créneaux pour la piscine.
%\end{itemize}
%\paragraph{Points faibles}
%\begin{itemize}
%  \item Parfois trop de monde,
%  \item Hors du campus,
%  \item Tu peux être considéré comme « étudiant extérieur » à la fac, et donc non-prioritaire sur certains sports.
%\end{itemize}
%
%\subsection{Associations culturelles étudiantes}\trad
%\subsubsection{À l'École Centrale}
%\paragraph{AED} Association des Élèves en Doctorat de l'École Centrale Nantes. Elle fédère tous les doctorants présents sur le site de l'ECN et organise des événements tant festifs que scientifiques. Rejoins-nous sur le groupe Facebook « AED-ECN » pour voir le fil des infos. Site internet : \url{http://website.ec-nantes.fr/aed/}
%\paragraph{AMAP Ecole Centrale} Le BDS et Centrale Vert de l'ECN propose à la rentrée 2011 des paniers de légumes bio, distribués toutes les semaines. Tu auras de plus amples informations à la rentrée 2011-2012, ou par mail.
%
%D'autre part, les bureaux des élèves (BDE, BDA, BDS) organisent en Septembre une soirée de rentrée des associations. Tu pourras y découvrir tous les clubs et éventuellement t'y inscrire ! Renseigne-toi sur la date.
%
%\subsubsection{Sur le campus universitaire}
%\paragraph{Association « Des pieds dans la tête »} Propose des ciné-philo ; le but de l'association est de sortir la philosophie du cadre de l'enseignement et d'ouvrir des débats plus larges par exemple sur la liberté. Site : \url{http://depidalate.free.fr/}
%\paragraph{Maison des Jeux} Tous les mercredis midi au TU, tu pourras découvrir et participer à des jeux de société et jeux en bois.

