\section{Sport and culture on campus}
\subsection {Where do sports?}
Wanna romp? Out of your 10 m \ {2} textsuperscript office?
Will make the sport! Three structures allow you that.

\subsubsection{The Athletic Association Centrale Nantes (AS)}
It is the association managed by the Sports Office (BDS) students from Central. As students, doctoral students may enroll in AS.
\paragraph{Contacts}
\begin{itemize}
  \item The email address of the office of sports: \ texttt {} bds@ec-nantes.fr
  \item Bureau 1 \ {st} textsuperscript floor of L
  \item \ texttt {} http://bds.campus.ec-nantes.fr
\end {itemize}
\paragraph{How to apply}
\begin{itemize}
  \item A check \ EUR {55},
  \item A medical certificate \ footnote {-filled meadows certificates are available at the office of BDS} non-cons-indication (if you specify the competition)
\end{itemize}

\paragraph{Highlights}
\begin{itemize}
  \item [$+$] Many sports (17)
  \item [$+$] Possibility of academic competitions,
  \item [$+$] Diversity of levels,
  \item [$+$] Some offer sports training with teacher
  \item [$+$] You may find yourself playing with your own students!
\end {itemize}
\paragraph{Weaknesses}
\begin{itemize}
  \item [$-$] You may find yourself playing with your own students,
  \item [$-$] Sometimes too crowded,
  \item [$-$] Requires consistency in training.
\end{itemize}

\paragraph{Warning} The competitions are usually held on Thursday afternoon (often at the same time as seminars).
As drives, they require a special commitment vis-à-vis your team and your opponents.
Before you commit, although check your obligations doctoral allow you to regularly participate in these competitions.

\paragraph{Some events throughout the year}
\begin{itemize}
  \item Outside Central: The Inter-Central, Challenge Lyon, BER (Brest), CAS (Supélec Rennes), the TOSS (Supélec Paris).
  \item In Central: Tournament 4 balls (T4B) 3 rackets tournament (T3R), Inter-groups.
\end{itemize}

\subsubsection {The Association Sportive du Personnel (ASP)}
It is the staff association of the Ecole Centrale (teachers, administrators, technicians, etc.. PhD and!). The association offers only four sports (futsal, tennis, gym and badminton) but asks only change. If you are motivated to get a sport, especially not hesitate! FYI, before there was table tennis and dance.

\paragraph{} 12 Slots - 13:30.
\paragraph{Contacts}
\begin {itemize}
  \item Chairperson: \ texttt {jean-paul.bouganne @ ec-nantes.fr} (02 40 37 25 92)
  \item Intranet site of ECN
\end {itemize}
\paragraph{How to apply}
\begin{itemize}
  \item A check \ EUR {13} (+ \ EUR {13} for tennis \ footnote {The tennis requires to pay a license FFT}).
  \item A medical certificate of non-cons-indication.
\end{itemize}

\paragraph{Highlights}
\begin{itemize}
  \item Cheap,
  \item Often fewer people
  \item Ideal to go romp without headaches
\end{itemize}
\paragraph{} Weaknesses
\begin{itemize}
  \item Few sports (4),
  \item No competition possible (except for tennis!)
\end{itemize}

\subsubsection {The SUAPS}
SUAPS is the agency responsible for sports in college Nantes.
This is a BIG structure based in the gym in front of the restaurant U.
Even if you are not a student of the college, you can still sign up at SUAPS.
This is usually a good alternative if the sport you desire is not available on the campus of Central.

\paragraph{Warning} If, as a PhD student, you are not enrolled at the university, you will be considered "foreign student."
Registration will cost more (\ EUR {50} instead of \ EUR {35}), and some sports "popular" you are not allowed.
Visit the website for more information SUAPS.

\paragraph{Highlights}
\begin{itemize}
  \item Many sports (47), including sports "rare"
  \item Many hardware
  \item Large variety of levels,
  \item Provides slots for the pool.
\end{itemize}
\paragraph{Weaknesses}
\begin{itemize}
  \item Sometimes too crowded,
  \item Off campus,
  \item You can be considered "foreign student" in college, and therefore not a priority in some sports.
\end{itemize}

\subsection {Cultural Associations Student}
\subsubsection {At the Ecole Centrale}
\paragraph{AED} Association of Students in PhD from Ecole Centrale Nantes. It brings together all the docs on the site of the ECN and organizes events as festive as scientists. Join us on Facebook group "ECN-AED" to see the thread information. Website: \ url {} http://website.ec-nantes.fr/aed/
\paragraph{AMAP Ecole Central} BDS and Central Green ECN offer in September 2011 baskets of organic vegetables, delivered weekly. Get more information at the start of 2011-2012, or email.

On the other hand, the offices of students (BDE, BDA, BDS) organizing an evening in September retraction associations. You can discover all the clubs and eventually you register! Learn about the date.

\subsubsection{On campus}
\paragraph{Association "Feet in the head"} Provides film-philosophy, the aim of the association is to remove part of the philosophy of education and to open wider debates on such freedom. Site: \ url {} http://depidalate.free.fr/
\paragraph{House of Games} Every Wednesday noon at TU, you can discover and participate in games and wooden games.

\section{Sport et culture sur le campus}\trad
%\todo{Vérifier les infos de cette section}
%\subsection{Où faire du sport ?}\trad
%Envie de te défouler ? De sortir de tes 10 m\textsuperscript{2} de bureau ?
%Va faire du sport ! Trois structures te permettent cela.
%
%\subsubsection{L'Association sportive de Centrale Nantes (AS)}
%C'est l'association gérée par le Bureau Des Sports (BDS) des étudiants de Centrale. En tant qu'étudiants, les doctorants peuvent s'inscrire à l'AS.
%\paragraph{Contacts}
%\begin{itemize}
%  \item L'adresse courriel du bureau des sports : \texttt{bds@ec-nantes.fr}
%  \item Bureau au 1\textsuperscript{er} étage du bâtiment L
%  \item \texttt{http://bds.campus.ec-nantes.fr}
%\end{itemize}
%\paragraph{Modalités d'inscription}
%\begin{itemize}
%  \item Un chèque de \EUR{55},
%  \item Un certificat médical\footnote{Des certificats prés-remplis sont disponibles au bureau du BDS} de non-contre-indication (précisez si vous faites de la compétition)
%\end{itemize}
%
%\paragraph{Points forts}
%\begin{itemize}
%  \item[$+$] Beaucoup de sports (17),
%  \item[$+$] Possibilité de faire des compétitions universitaires,
%  \item[$+$] Diversité de niveaux,
%  \item[$+$] Certains sports proposent des entrainements avec professeur,
%  \item[$+$] Tu peux te retrouver à jouer avec tes propres élèves !
%\end{itemize}
%\paragraph{Points faibles}
%\begin{itemize}
%  \item[$-$] Tu peux te retrouver à jouer avec tes propres élèves,
%  \item[$-$] Parfois trop de monde,
%  \item[$-$] Nécessite une régularité dans les entraînements.
%\end{itemize}
%
%\paragraph{Attention} Les compétitions ont généralement lieu les jeudis après-midi (souvent en même temps que des séminaires).
%Tout comme les entraînements, elles demandent un engagement particulier vis-à-vis vos équipes et de vos adversaires.
%Avant de t'engager, vérifie bien que tes obligations de doctorants te permettent de participer régulièrement à ces compétitions.
%
%\paragraph{Quelques événements tout au long de l'année}
%\begin{itemize}
%  \item Hors de Centrale : Les Inter-Centrales, le Challenge Lyon, le TEB (Brest), les CAS (Supélec Rennes), le TOSS (Supélec Paris).
%  \item Dans Centrale : Tournoi des 4 ballons (T4B), tournoi des 3 raquettes (T3R), Inter-groupes.
%\end{itemize}
%
%\subsubsection{L'Association Sportive du Personnel (ASP)}
%C'est l'association du personnel de l'École Centrale (professeurs, administratifs, techniciens, etc. et doctorants !). L'association ne propose que quatre sports (foot en salle, Tennis, musculation et badminton) mais ne demande qu'à évoluer. Si tu es motivé pour monter un sport, n'hésite surtout pas ! Pour info, il y avait avant du ping-pong et de la danse.
%
%\paragraph{Créneaux horaires} 12h--13h30.
%\paragraph{Contacts}
%\begin{itemize}
%  \item Président : \texttt{jean-paul.bouganne@ec-nantes.fr} (02 40 37 25 92)
%  \item Intranet du site de l'ECN
%\end{itemize}
%\paragraph{Modalités d'inscription}
%\begin{itemize}
%  \item Un chèque de \EUR{13} (+ \EUR{13} pour le tennis\footnote{Le tennis nécessite de payer une licence FFT}).
%  \item Un certificat médical de non-contre-indication.
%\end{itemize}
%
%\paragraph{Points forts}
%\begin{itemize}
%  \item Pas cher,
%  \item Souvent moins de monde,
%  \item Idéal pour aller se défouler sans prise de tête
%\end{itemize}
%\paragraph{Points faibles}
%\begin{itemize}
%  \item Peu de sports (4),
%  \item Pas de compétition possible (sauf pour le tennis !)
%\end{itemize}
%
%\subsubsection{Le SUAPS}
%Le SUAPS est l'organisme responsable des sports à la fac de Nantes.
%C'est une GROSSE structure basée dans le gymnase en face du resto U.
%Même si tu n'es pas étudiant de la fac, tu peux quand même t'inscrire au SUAPS.
%C'est en général une bonne alternative si le sport que tu désires n'est pas disponible sur le campus de Centrale.
%
%\paragraph{Attention} Si, en tant que doctorant, tu n'es pas inscrit à l'université, tu seras considéré comme « étudiant extérieur ».
%L'inscription te coûtera plus cher (\EUR{50} au lieu de \EUR{35}), et certains sports « populaires » ne te seront pas autorisés.
%Rendez-vous sur le site du SUAPS pour plus d'informations.
%
%\paragraph{Points forts}
%\begin{itemize}
%  \item Beaucoup de sports (47), dont des sports « rares »,
%  \item Beaucoup de matériel,
%  \item Grande diversité de niveaux,
%  \item Propose des créneaux pour la piscine.
%\end{itemize}
%\paragraph{Points faibles}
%\begin{itemize}
%  \item Parfois trop de monde,
%  \item Hors du campus,
%  \item Tu peux être considéré comme « étudiant extérieur » à la fac, et donc non-prioritaire sur certains sports.
%\end{itemize}
%
%\subsection{Associations culturelles étudiantes}\trad
%\subsubsection{À l'École Centrale}
%\paragraph{AED} Association des Élèves en Doctorat de l'École Centrale Nantes. Elle fédère tous les doctorants présents sur le site de l'ECN et organise des événements tant festifs que scientifiques. Rejoins-nous sur le groupe Facebook « AED-ECN » pour voir le fil des infos. Site internet : \url{http://website.ec-nantes.fr/aed/}
%\paragraph{AMAP Ecole Centrale} Le BDS et Centrale Vert de l'ECN propose à la rentrée 2011 des paniers de légumes bio, distribués toutes les semaines. Tu auras de plus amples informations à la rentrée 2011-2012, ou par mail.
%
%D'autre part, les bureaux des élèves (BDE, BDA, BDS) organisent en Septembre une soirée de rentrée des associations. Tu pourras y découvrir tous les clubs et éventuellement t'y inscrire ! Renseigne-toi sur la date.
%
%\subsubsection{Sur le campus universitaire}
%\paragraph{Association « Des pieds dans la tête »} Propose des ciné-philo ; le but de l'association est de sortir la philosophie du cadre de l'enseignement et d'ouvrir des débats plus larges par exemple sur la liberté. Site : \url{http://depidalate.free.fr/}
%\paragraph{Maison des Jeux} Tous les mercredis midi au TU, tu pourras découvrir et participer à des jeux de société et jeux en bois.
